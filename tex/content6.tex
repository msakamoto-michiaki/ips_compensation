\subsection{例題3:偏光子・A-plate・LC の光軸角度が CR 等高線に与える影響(\texttt{ex3.py})}
本例題では,A-plate および C-plate の膜厚を固定した状態で,入力側スタック(偏光子・A-plate・LC)の方位角 $\alpha$ を微小に回転させた場合に,暗状態コントラスト分布 $\mathrm{CR}(\theta,\phi)$ の等高線形状(ISO\_PLOT)がどのように変化するかを検討した.
上側スタック(解析側)は固定し,下側スタック(入力側)のみを回転させる設定である.
出力として,等高線画像とともに,$\log_{10}\mathrm{CR}$ の方位方向標準偏差の平均値
\[
D = \langle \mathrm{std}_{\phi}(\log_{10}\mathrm{CR}) \rangle_{\theta}
\]
を「方位異方性指標」として算出した.

\subsubsection{角度パラメータの関係と定義}
下側スタック(入力側)の各光軸は,偏光子角度 \texttt{pol\_in} に追従して回転する.
各層の光軸方位角は次式で定義される:

\[
\begin{aligned}
\alpha_{\mathrm{pol,in}} &= \texttt{pol\_in},\\
\alpha_{A,\mathrm{bot}} &= \texttt{pol\_in} + \texttt{relA},\\
\alpha_{\mathrm{LC}} &= \texttt{pol\_in} + \texttt{lc\_rel},\\
\alpha_{A,\mathrm{top}} &= 90^\circ + \texttt{pol\_out} + \texttt{relA},
\end{aligned}
\]
ここで $\texttt{pol\_out}=0$ のため,上側スタック(top)は固定され,入力側(bot)が回転する.
本例題では \texttt{relA}=0.25°, \texttt{lc\_rel}=0.5° であり,したがって
\[
\alpha_{A,\mathrm{bot}}=\texttt{pol\_in}+0.25^\circ,\qquad
\alpha_{\mathrm{LC}}=\texttt{pol\_in}+0.5^\circ,\qquad
\alpha_{A,\mathrm{top}}=90.25^\circ.
\]

\subsubsection{評価手順}
C-plate を除いた構成($C_{\mu\mathrm{m}}=0$)で $A_{\mathrm{scale}}=1.50$ に固定し,
$\texttt{pol\_in} \in \{0.0, 0.2, 0.5, 1.0, 2.0\}$ [deg] を走査した.
白色(B/G/R 平均),分散整合(matched)条件で $\theta_{\max}=60^\circ$,$\Delta\theta=\Delta\phi=5^\circ$ の格子上で $\mathrm{CR}(\theta,\phi)$ を算出し,
しきい値超過率 $\mathrm{frac}\{\mathrm{CR}>100\}$ と異方性指標 $D$ を求めた.

\subsubsection{結果:ISO\_PLOT の非対称化と異方性指標}
$\texttt{pol\_in}=0^\circ$ の場合,ISO\_PLOT は上下左右で対称な四葉形を示し,方位対称性が保たれている.
しかし $\texttt{pol\_in}$ を増加させると,$\phi$ 方向の明暗が偏り,等高線が一部方位で押しつぶされるような\textbf{非対称性}が現れた.
これは入力スタックの回転により偏光状態の回転とリタデーションが非対称化し,特定方位の暗状態漏れが増加するためである.
この結果,ISO 等高線の外形が $\phi$ に依存して歪み,暗状態維持範囲に方位的偏りが生じる。

表\ref{tab:ex3_distortion_fixed}に結果を示す.ここで $D$ は「方位異方性(振幅差)」を示すものであり,**非対称性そのものではない**点に注意されたい.
対称的な四葉形($\texttt{pol\_in}=0$)では $\phi$ 方向の強弱が最も大きく,$\mathrm{CR}$ の $\phi$ 変化が強いため $D$ が最大となる.
一方,$\texttt{pol\_in}$ の増加により全体の振幅差がやや緩和されるため $D$ は減少するが,ISO 図では逆に\textbf{非対称性(形状の傾き)が増加}する.
このことから,$D$ は「異方性の強さ」を表す指標であり,「非対称の度合い」を表すものではない.

\begin{table}[t]
  \centering
  \caption{例題3:$\texttt{pol\_in}$ に対するしきい値超過率と異方性指標(matched,$A_{\mathrm{scale}}=1.50$,$C_{\mu\mathrm{m}}=0$).}
  \label{tab:ex3_distortion_fixed}
  \begin{tabular}{c|cc}
    \hline
    \texttt{pol\_in} [deg] & $\mathrm{frac}\{\mathrm{CR}>100\}$ & $D=\langle \mathrm{std}_\phi(\log_{10}\mathrm{CR})\rangle_\theta$ \\
    \hline
    0.0 & 0.340 & 0.621 \\
    0.2 & 0.336 & 0.608 \\
    0.5 & 0.334 & 0.581 \\
    1.0 & 0.347 & 0.537 \\
    2.0 & 0.342 & 0.468 \\
    \hline
  \end{tabular}
\end{table}

\subsubsection{設計上の含意と今後の拡張}
ISO\_PLOT と $D$ を併用することで,
(1) 方位依存の「異方性強度(D)」と,
(2) 方位対称性の崩れ(非対称形状)
を区別して議論できる.
実際の設計では,組立て誤差や貼り合わせ角度差によって入力側の $\alpha$ が微小にずれ,暗状態の非対称化を引き起こすため,
\textbf{ISO 図による方位依存の可視化}と,\textbf{追加指標(例えば $\phi$ と $\phi+\pi$ の差分標準偏差)}による非対称性の定量化が重要である.

% \begin{figure}[t]
%   \centering
%   % \includegraphics[width=0.48\linewidth]{ISO_contour_pol_in_0.00deg_C+0.00um_A1.50.png}
%   % \includegraphics[width=0.48\linewidth]{ISO_contour_pol_in_2.00deg_C+0.00um_A1.50.png}
%   \caption{例題3:pol\_in 掃引による ISO 等高線の変化.上:pol\_in=0°, 下:2°.上側スタック固定・下側スタック回転条件.}
%   \label{fig:ex3_iso_pol_in}
% \end{figure}
