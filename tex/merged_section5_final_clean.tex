\documentclass[11pt,a4paper]{article}

% ----------- Packages -----------
\usepackage[margin=1in]{geometry}
\usepackage{amsmath,amssymb,mathtools}
\usepackage{bm}
\usepackage{physics}   % \dv, \pdv, etc.
\usepackage{siunitx}
\usepackage{graphicx}
% (Removed hyperref per user's request to avoid compile issues)

% ----------- Macros (avoid names used by 'physics') -----------
\newcommand{\qvec}{\mathbf{q}}      % transverse position (x,y)
\newcommand{\pvec}{\mathbf{p}}      % transverse canonical momentum (p_x,p_y)
\newcommand{\rr}{\mathbf r}
\newcommand{\nn}{\mu}               % refractive index (book uses \mu)
\newcommand{\Hz}{H}                 % reduced Hamiltonian (z as time)
\newcommand{\HJ}{H_0}               % 3D constrained Hamiltonian
\newcommand{\Id}{\mathrm{I}}        % identity matrix symbol
\newcommand{\Jmat}{\begin{pmatrix} 0 & \Id \\ -\Id & 0 \end{pmatrix}}
\newcommand{\Lie}{\mathcal{L}}      % Lie operator
\newcommand{\pL}{\mathbf{p}_L}      % momentum from the z-Lagrangian

\title{Geometrical Optics via Hamiltonian and Lie Methods:\\
From Fermat's Principle to ABCD, Poisson Brackets, and Aberrations}
\author{}
\date{}

\begin{document}
\maketitle

\begin{abstract}
We derive the Hamiltonian formulation of geometrical optics from Fermat's principle, reduce it with the longitudinal coordinate $z$ as the evolution parameter, perform the paraxial expansion, and connect to the ABCD formalism. Interfaces appear as canonical kicks generated by surface functions. We then give a detailed exposition of symplecticity, the Poisson bracket and its role, canonical maps, and Lie transforms (including BCH/Magnus viewpoints). Finally, we show how this framework extends systematically to higher-order aberrations while preserving invariants such as etendue, provide an explicit one-dimensional worked example, and derive Seidel aberrations directly with Lie transforms.
\end{abstract}

% ---------------------------------------
\section{Fermat's Principle and Canonical Variables}
Let $\rr=(x,y,z)$, $\hat{\mathbf s}=\dv{\rr}{s}$ with arclength $s$, and refractive index $\nn(\rr)$.
Fermat's action is $\mathcal S=\int \nn\,ds$; using an arbitrary parameter $\lambda$,
\begin{equation}
L(\rr,\dot{\rr})=\nn(\rr)\,\lVert \dot{\rr}\rVert,\qquad
\mathbf p_{3\mathrm{D}}:=\pdv{L}{\dot{\rr}}=\nn\,\frac{\dot{\rr}}{\lVert \dot{\rr}\rVert}
=\nn\,\hat{\mathbf s},
\end{equation}
so $\lVert \mathbf p_{3\mathrm{D}}\rVert=\nn$.
The eikonal $S(\rr)$ satisfies along rays $dS=\nn\,ds=\mathbf p_{3\mathrm{D}}\!\cdot d\rr$, hence
\begin{equation}
\nabla S=\mathbf p_{3\mathrm{D}},\qquad \lVert \nabla S\rVert=\nn(\rr).
\end{equation}

% ---------------------------------------
\section{3D Constrained Hamiltonian}
On full $(\rr,\mathbf p_{3\mathrm{D}})$ space take
\begin{equation}
\HJ(\rr,\mathbf p_{3\mathrm{D}})=\frac12\!\left(\lVert \mathbf p_{3\mathrm{D}}\rVert^2-\nn(\rr)^2\right)=0.
\end{equation}
Hamilton's equations (with any parameter $\tau$) give the ray equations after choosing $ds/d\tau=\nn$:
\begin{equation}
\dv{\rr}{s}=\hat{\mathbf s},\qquad \dv{}{s}\big(\nn \hat{\mathbf s}\big)=\nabla \nn.
\end{equation}

% ---------------------------------------
\section{Reduction with $z$ as ``Time''}
Let $\qvec=(x,y)$ and split $\mathbf p_{3\mathrm{D}}=(\pvec,p_z)$ with
$p_z=+\sqrt{\nn(\qvec,z)^2-\lVert \pvec\rVert^2}$ for forward propagation.
The reduced Hamiltonian is
\begin{equation}
\boxed{\ \Hz(z,\qvec,\pvec)=-\,p_z=-\sqrt{\nn(\qvec,z)^2-\lVert \pvec\rVert^2}\ }.
\end{equation}
Hamilton's equations (now with ``time'' $z$) are
\begin{equation}
\dv{\qvec}{dz}=\pdv{\Hz}{\pvec}=\frac{\pvec}{\sqrt{\nn^2-\lVert \pvec\rVert^2}},
\qquad
\dv{\pvec}{dz}=-\pdv{\Hz}{\qvec}=\frac{\nn\,\nabla_{\qvec}\nn}{\sqrt{\nn^2-\lVert \pvec\rVert^2}}.
\end{equation}

\paragraph{Equivalence with the $z$-Lagrangian.}
With $q'=\dv{\qvec}{dz}$ and $ds=\sqrt{1+\lVert q'\rVert^2}\,dz$,
\begin{equation}
L(z,\qvec,q')=\nn(\qvec,z)\sqrt{1+\lVert q'\rVert^2},\qquad
\pvec=\pdv{L}{q'}=\nn\,\frac{q'}{\sqrt{1+\lVert q'\rVert^2}}.
\end{equation}
Thus $q'=\pvec/\sqrt{\nn^2-\lVert \pvec\rVert^2}$ and $H=\pvec\!\cdot q'-L$ reproduces $\Hz$.

% ===== Step-by-step Legendre algebra (explicit) =====
\subsection*{Legendre transform algebra (step-by-step)}
Let $\pL$ denote the momentum defined from the $z$-Lagrangian ($\pL\equiv\pvec$):
\begin{equation}
\pL=\pdv{L}{q'}=\nn\,\frac{q'}{\sqrt{1+\lVert q'\rVert^2}}.
\label{eq:pLdef}
\end{equation}
Taking norms in \eqref{eq:pLdef}:
\[
\lVert\pL\rVert^2=\nn^2\,\frac{\lVert q'\rVert^2}{1+\lVert q'\rVert^2}
\Rightarrow
\lVert q'\rVert^2=\frac{\lVert\pL\rVert^2}{\nn^2-\lVert\pL\rVert^2},
\quad
\sqrt{1+\lVert q'\rVert^2}=\frac{\nn}{\sqrt{\nn^2-\lVert\pL\rVert^2}}.
\]
Hence
\begin{equation}
\boxed{\ q'=\frac{\pL}{\sqrt{\nn^2-\lVert\pL\rVert^2}}\ },\qquad
\boxed{\ \pL\!\cdot q'=\frac{\lVert\pL\rVert^2}{\sqrt{\nn^2-\lVert\pL\rVert^2}}\ },\qquad
\boxed{\ L=\frac{\nn^2}{\sqrt{\nn^2-\lVert\pL\rVert^2}}\ }.
\end{equation}
Therefore
\[
H=\pL\!\cdot q'-L
=\frac{\lVert\pL\rVert^2-\nn^2}{\sqrt{\nn^2-\lVert\pL\rVert^2}}
= -\,\sqrt{\nn^2-\lVert\pL\rVert^2}
= -p_z .
\]

% ---------------------------------------
\section{Paraxial Expansion and ABCD}
For $\lVert \pvec\rVert\ll \nn$,
\begin{equation}
\Hz=-\sqrt{\nn^2-\lVert \pvec\rVert^2}
= -\,\nn+\frac{\lVert \pvec\rVert^2}{2\nn}+\mathcal O(\lVert \pvec\rVert^4).
\end{equation}

\subsection{Uniform Medium ($\nn=n_0$)}
Quadratic Hamiltonian: $\Hz^{(2)}=\lVert \pvec\rVert^2/(2n_0)$, so
\begin{equation}
\dv{\qvec}{dz}=\frac{\pvec}{n_0},\qquad \dv{\pvec}{dz}=0
\ \Rightarrow\
\binom{\qvec'}{\pvec'}=
\begin{pmatrix}\Id & \tfrac{L}{n_0}\Id\\ 0&\Id\end{pmatrix}
\binom{\qvec}{\pvec}.
\end{equation}

\subsection{Quadratic GRIN}
If $n^2\simeq n_0^2\!\left(1-g^2\lVert \qvec\rVert^2\right)$, then
\begin{equation}
\Hz^{(2)}\simeq \frac{\lVert \pvec\rVert^2}{2n_0}-\frac{n_0g^2}{2}\lVert \qvec\rVert^2,
\end{equation}
giving a harmonic channel with ABCD matrix
\begin{equation}
M_{\mathrm{GRIN}}(L)=
\begin{pmatrix}
\cos(gL) & \tfrac{1}{n_0 g}\sin(gL)\\
-\,n_0 g\sin(gL) & \cos(gL)
\end{pmatrix}.
\end{equation}

% ---------------------------------------
\section{Interfaces as Canonical Kicks}
\paragraph{Refraction as a canonical ``kick''.}
We model a refracting interface as an instantaneous canonical map that keeps position continuous and changes momentum:
\begin{equation}
\qvec^+=\qvec^- ,\qquad \pvec^+ \neq \pvec^- .
\end{equation}
Use a type-2 generating function $G(\qvec,\mathbf P)=\qvec\!\cdot\!\mathbf P + W_{\mathrm{surf}}(\qvec)$, which yields
\begin{equation}
\qvec^+=\qvec^- ,\qquad \pvec^+=\pvec^- - \nabla_{\qvec} W_{\mathrm{surf}}(\qvec^-).
\end{equation}

\paragraph{Snell from eikonal continuity.}
Let the surface be $z=f(\qvec)$ with small slope. Eikonal continuity across the surface normal gives, in the paraxial limit,
\begin{equation}
\Delta \pvec := \pvec^+ - \pvec^- = -\,(\mu_{\mathrm{after}}-\mu_{\mathrm{before}})\, \nabla_{\qvec} f(\qvec).
\end{equation}

\paragraph{Spherical surface.}
For a spherical interface of radius $R$ (convex toward $+z$), $f(\qvec)\simeq \lVert \qvec\rVert^2/(2R)$, hence
\begin{equation}
\Delta \pvec = -\,\frac{\mu_{\mathrm{after}}-\mu_{\mathrm{before}}}{R}\,\qvec \equiv -\,\Phi\,\qvec,\qquad \Phi:=\frac{\mu_{\mathrm{after}}-\mu_{\mathrm{before}}}{R}.
\end{equation}

\paragraph{Surface generator and thin-map.}
Matching $\pvec^+=\pvec^- - \nabla_{\qvec} W_{\mathrm{surf}}$ gives $\nabla_{\qvec} W_{\mathrm{surf}}=\Phi\,\qvec$, thus
\begin{equation}
W_{\mathrm{surf}}(\qvec)=\frac{\Phi}{2}\,\lVert \qvec\rVert^2.
\end{equation}
Therefore the thin-interface canonical map is
\begin{equation}
\binom{\qvec^+}{\pvec^+}=
\begin{pmatrix}\Id&0\\ -\Phi \Id&\Id\end{pmatrix}
\binom{\qvec^-}{\pvec^-},
\end{equation}
which is symplectic and preserves etendue.

% ---------------------------------------
\section{Symplecticity, Poisson Brackets, Canonical Maps, and Lie Transforms (Detailed)}
\subsection*{What ``symplectic'' means and why optics has it}
Let $\mathbf z=(\qvec,\pvec)^\top\in\mathbb R^{2n}$ and
$J=\begin{psmallmatrix}0&\Id\\ -\Id&0\end{psmallmatrix}$.
A differentiable map $T:\mathbf z\mapsto \mathbf z'$ is \emph{symplectic} (canonical) iff its Jacobian $M=\pdv{\mathbf z'}{\mathbf z}$ satisfies
\begin{equation}
M^\top J M = J .
\label{eq:sympCond}
\end{equation}
Equivalently, $T$ preserves the two-form $\omega=\sum_i dq_i\wedge dp_i$.
Hamiltonian flows $\dot{\mathbf z}=J\nabla H$ are symplectic because
$\mathcal L_{X_H}\omega=0$; instantaneous ``kicks'' generated by a scalar $W$
via $\delta f=\{f,W\}$ are also symplectic.

\subsection*{Poisson bracket: definition and identities}
On $\mathbb R^{2n}$ with canonical coordinates $\mathbf z=(q_1,\dots,q_n,p_1,\dots,p_n)$,
\begin{equation}
\{f,g\}:=\sum_{i=1}^n\left(\pdv{f}{q_i}\pdv{g}{p_i}-\pdv{f}{p_i}\pdv{g}{q_i}\right)
=(\nabla f)^\top J (\nabla g),
\end{equation}
bilinear, antisymmetric, and satisfying the Leibniz and Jacobi identities.

\subsection*{Canonical transformations $\Leftrightarrow$ Poisson preservation}
A diffeomorphism $T$ is canonical iff
\begin{equation}
\{f,g\}\circ T=\{f\circ T,\,g\circ T\}\quad(\forall f,g),
\end{equation}
equivalently $M^\top J M=J$.

\subsection*{Lie operator and finite Lie transform}
Define $\Lie_W f:=\{f,W\}$.
Then $T=\exp(\Lie_W)$ is canonical; products are canonical, and with $z$-dependence
\begin{equation}
T=\mathcal T\exp\!\Big(\int \Lie_{H(z)}\,dz\Big)
\end{equation}
for continuous media.

\subsection*{BCH/Magnus viewpoint}
\begin{equation}
e^{\Lie_A}e^{\Lie_B}=e^{\Lie_{A\oplus B}},\quad
A\oplus B=A+B+\tfrac12\{A,B\}+\cdots .
\end{equation}

\subsection*{Relation to ABCD and $\mathfrak{sp}$}
Drift and thin-surface matrices arise from quadratic generators and span $\mathfrak{sp}(2,\mathbb R)$; linear optics is a product of such exponentials.

\subsection*{Are all canonical maps Lie transforms?}
Narrowly (single time-independent exponential) no; broadly (time-ordered, products/BCH) yes for maps connected to identity.

\subsection*{Kicks vs.\ flows; discontinuities at refractions}
Refractions are kicks $T_{\rm surf}=e^{\Lie_{W_{\rm surf}}}$; flows from $H(z)$; both canonical.

\subsection*{Systematic higher orders: effective cubic generator}
With $W=W_2+W_3+W_4+\cdots$, up to third order
\begin{equation}
\mathbf z_{\rm out}=M\,\mathbf z_{\rm in}+J\,\nabla W_{3,\mathrm{eff}}(M\,\mathbf z_{\rm in}),
\end{equation}
where $W_{3,\mathrm{eff}}(\mathbf Z)=\sum_e W_{3,e}(M_e\mathbf Z)$.

% ---------------------------------------
\section{Seidel Aberrations via Lie Transforms (Explicit Derivation)}
We work in the image-side canonical variables $\mathbf Z=(X,Y,P_X,P_Y)^\top$.
For an axisymmetric system, the effective cubic generator can be written using $O(2)$-invariant monomials:
\begin{align}
\rho^2&:=X^2+Y^2,\qquad \upsilon^2:=P_X^2+P_Y^2,\qquad
\chi:=X P_X+Y P_Y,\qquad
L:=X P_Y-Y P_X .
\end{align}
Up to total degree three, a minimal invariant basis is
\begin{equation}
\boxed{\ W_{3,\mathrm{eff}} = a\,\rho^2\chi\;+\;b\,\chi^3\;+\;c\,\rho^2\upsilon\;+\;d\,\upsilon\chi\;+\;e\,\chi\,L^2\ }.
\label{eq:W3effSeidel}
\end{equation}
Here $a,b,c,d,e$ are the \emph{system Seidel sums} (dimensioned coefficients) determined by element-wise contributions transported to the image side and added.

\paragraph{Third-order image-space map.}
The correction to the linear image $M\mathbf z$ is
\begin{equation}
\Delta\mathbf Z = J\,\nabla W_{3,\mathrm{eff}}(\mathbf Z)
=
\begin{pmatrix}
\partial_{P_X} W_{3,\mathrm{eff}}\\[2pt]
\partial_{P_Y} W_{3,\mathrm{eff}}\\[2pt]
-\partial_X W_{3,\mathrm{eff}}\\[2pt]
-\partial_Y W_{3,\mathrm{eff}}
\end{pmatrix}.
\end{equation}
Differentiating \eqref{eq:W3effSeidel} and sorting by field ($\rho$) and aperture ($\upsilon$) powers yields the classical Seidel partition:
\begin{align}
\text{Spherical (}S\text{)}&:\ \ \propto d\,\upsilon^2\,\mathbf U \ +\ 3b\,\chi^2\,\mathbf U,
\\
\text{Coma (}C\text{)}&:\ \ \propto (c+2a)\,\rho^2\,\mathbf U\;+\; (d+3b)\,\chi\,\rho\,\hat{\boldsymbol\rho},
\\
\text{Astig./Field Curv. (}A,F\text{)}&:\ \ \propto a\,\rho^2\,\hat{\boldsymbol\rho}\;+\;c\,\rho^2\,\mathbf U,
\\
\text{Distortion (}D\text{)}&:\ \ \propto e\,L^2\,\hat{\boldsymbol\rho},
\end{align}
where $\mathbf U=(P_X,P_Y)$ and $\hat{\boldsymbol\rho}=(X,Y)$.
Thus $(a,b,c,d,e)$ map one-to-one to the five Seidel aberrations
$\{S,C,A,F,D\}$.

\paragraph{Element-wise coefficients.}
For a system composed of continuous segments ($H_3$) and surfaces ($W_{3,\rm surf}$), each element $e$ contributes coefficients $(a_e,\dots,e_e)$ computed from its local cubic generator in its own coordinates, then \emph{transported} to the image side by the downstream linear matrix $M_e$:
\[
W_{3,e}^{\rm (img)}(\mathbf Z)=W_{3,e}(M_e\mathbf Z)
\quad\Rightarrow\quad
(a,\dots,e)=\sum_e (a_e,\dots,e_e)\cdot \mathcal{T}(M_e),
\]
with $\mathcal{T}(M_e)$ the linear mixing induced by $M_e$ on the basis
$\{\rho^2\chi,\chi^3,\rho^2\upsilon,\upsilon\chi,\chi L^2\}$.
Typical building blocks:
\begin{itemize}
\item \textbf{Uniform drift:} no cubic term ($H$ quadratic) $\Rightarrow$ zero contribution.
\item \textbf{Quadratic GRIN:} again purely quadratic $\Rightarrow$ zero cubic.
\item \textbf{Surface (thin spherical):} expanding Snell to third order yields a pure \emph{kick} $W_{3,\rm surf} = \alpha\,\rho^2\chi + \beta\,\chi^3$ in local variables, with $\alpha,\beta$ proportional to surface curvature and refractive jump; transporting by $M_e$ populates $(a,b)$ and, via mixing, also $(c,d,e)$ downstream.
\end{itemize}

\paragraph{How to compute $(a,b,c,d,e)$ algorithmically.}
\begin{enumerate}
\item Compute and store the \emph{cumulative} linear matrices $M_\rightarrow$ (from object to each element) and $M_\leftarrow$ (from each element to image).
\item For each element $e$, write its local cubic generator in canonical form $W_{3,e}(\mathbf z)=\tfrac{1}{6}\,T^{(e)}_{ijk}\,z_i z_j z_k$.
\item Transport to image side: $T^{(e)}_{\rm img}=M_\leftarrow^{\otimes 3}\,T^{(e)}\,(M_\rightarrow)^{\otimes 0}$ acting on indices.
\item Project $T^{(e)}_{\rm img}$ onto the invariant basis of \eqref{eq:W3effSeidel} to extract $(a_e,\dots,e_e)$.
\item Sum over $e$ to get $(a,\dots,e)$ and insert in $\Delta\mathbf Z=J\nabla W_{3,\rm eff}$.
\end{enumerate}
This gives the Seidel polynomial map while \emph{preserving symplecticity exactly}.

% ---------------------------------------
\section{Worked 1D Example: Two Thin Lenses with an Air Gap}
Consider two thin spherical lenses $A$ and $B$ (powers $\Phi_A,\Phi_B$) separated by an air gap $L$.
\paragraph{Quadratic (ABCD).}
With $K_X=\begin{psmallmatrix} 1&0\\ -\Phi_X&1\end{psmallmatrix}$ and $N=\begin{psmallmatrix}1&L\\0&1\end{psmallmatrix}$, the net matrix is
\begin{equation}
M=K_B N K_A=
\begin{pmatrix}
1 - L\Phi_A & L\\
-(\Phi_A+\Phi_B) + L\Phi_A\Phi_B & 1 - L\Phi_B
\end{pmatrix},\qquad \det M=1.
\end{equation}
\paragraph{Adding third order.}
Let the cubic generators at the lenses be $W_{3,A}=a_A x^3$ and $W_{3,B}=a_B x^3$.
Transport $W_{3,A}$ to the output with $M_A=K_B N$, so that
\begin{equation}
W_{3,\mathrm{eff}}(X,P)= a_A\,(m_{11}^{(A)}X+m_{12}^{(A)}P)^3 + a_B X^3,
\end{equation}
and then apply $\Delta z = J\nabla W_{3,\mathrm{eff}}$ to obtain the third-order correction on top of the linear image $M(x,p)$.

% ---------------------------------------
\section{Why this Framework is Useful}
\begin{itemize}
\item \textbf{Unity:} continuous propagation ($\Hz$ flow) and interfaces ($W_{\rm surf}$ kicks) live in the same canonical language and compose symplectically.
\item \textbf{Paraxial to nonparaxial:} keeping only $W_2$ and quadratic $\Hz$ reproduces ABCD; adding $W_3,W_4,\dots$ yields aberrations while preserving invariants (etendue, Helmholtz--Lagrange).
\item \textbf{Computation:} Lie series, BCH/Magnus expansions, and normal forms make multi-element systems tractable with guaranteed symplecticity.
\end{itemize}

% ---------------------------------------
\section*{Notation and Sign Conventions}
We use $\mu(\rr)$ (often $n$) for refractive index. The canonical variables in the reduced (design) formulation are $(\qvec,\pvec)=(x,y,p_x,p_y)$; the longitudinal momentum is $p_z=+\sqrt{\mu^2-\lVert\pvec\rVert^2}$ (forward propagation). For a spherical refracting interface with radius $R$,
$\Phi=(\mu_{\mathrm{after}}-\mu_{\mathrm{before}})/R$; the corresponding surface matrix is
$\begin{psmallmatrix}\Id&0\\ -\Phi \Id&\Id\end{psmallmatrix}$.
Etendue conservation follows from symplecticity.

\end{document}
