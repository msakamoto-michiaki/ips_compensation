\section{光学補償最適化プログラムの概要}
\label{sec:opt_case1_case2}
本節では、偏光板(クロスニコル)下の IPS 系暗状態に対し、補償板(A-plate, C-plate)を付加したときの
コントラスト比(CR)改善を目的とした光学補償設計の最適化プログラムの概要を示す。

\subsection{入力}
\subsubsection{構成}
対象とするスタックは簡便のため,図~\ref{fig_stack}に示すように、LC/A/Cとし、
入射側偏光板 (Polarizer; POL)の吸収軸(${\bm c_1}$)とLCの屈折率の長軸が平行のLC吸収軸基準(Eモード)
に配置する。
出射側偏光板 (Analyzer; ANA)の吸収軸(${\bm c_2}$)とPOLの吸収軸${\bm c_1}$は直交するクロスニコル配置とする.
尚、通常のPOLの透過率とLCの屈折率の長軸が平行なLC透過軸基準(Oモード)の場合はC/A/LCのスタックでEモードと同様の
結果が得られる。
実際には光学フィルムをC-plate($R_{th}$)/A-plate($R_{0}$)の組み合わせで記述され、一般には
C/A/LC/A/Cスタックでモデル化される。
しかし、光学補償の設計指針はLC/A/Cスタックと同様であり、CR視野角の定性的な傾向も同様のため、
本スタックで実施する。

\subsubsection{膜厚}
本実装では、等方屈折率を $n_o=\mathrm{NO\_BASE}$ とし、各層の複屈折を $\Delta n$ として
異常屈折率 $n_e=n_o+\Delta n$ を与える。
位相差(リタデーション)$\mathrm{Re}$ と膜厚 $d$ の関係は
\begin{equation}
  \mathrm{Re} = \Delta n \, d,
\end{equation}
であり、与えられた $\mathrm{Re}$(nm)と $\Delta n$ から膜厚は
\begin{equation}
  d\,[\mu\mathrm{m}] = \frac{\mathrm{Re}\,[\mathrm{nm}]}{1000\,\Delta n}
  \label{eq:d_from_Re}
\end{equation}
と換算できる。

LC は $d_{\mathrm{LC}}$ と $\Delta n_{\mathrm{LC}}$ を固定し、設計値として
$\mathrm{Re}_{\mathrm{LC}}=\Delta n_{\mathrm{LC}} d_{\mathrm{LC}}$ が定まる。
一方、A-plate は 主に面内リタデーション($R_0$)を補償する役割
C-plate は主に 厚み方向リタデーション($R_{th}$)を担い斜入射起因の残差を抑える役割を担う。
A-plate の膜厚は、選択した材料(upper/lower)の $\Delta n_A$ により $d_A=\mathrm{Re}_A/(1000\,\Delta n_A)$ として決まる。

\footnote{
プログラムの作成経緯から、実際にはスケール係数 $A_{\mathrm{scale}}$ により
\begin{equation}
 \mathrm{Re}_{A} = A_{\mathrm{scale}} \, \mathrm{Re}_{A,\mathrm{base}}
 \label{eq:ReA_scale}
\end{equation}
で与える。ここで $\mathrm{Re}_{A,\mathrm{base}}$ は「$\mathrm{Re}_{LC}/2$」に固定する必然性はなく、
(A/LC/A 構成の ``each'' 割当として $\mathrm{Re}_{LC}/2$ を用いる流儀は存在するが)、
本節の単一 A では目的関数に応じて $\mathrm{Re}_{A,\mathrm{base}}$ を別途与える(あるいは最適化変数として扱う)。
}

同様に、C-plate の設計は膜厚 $d_C$ を直接固定するのではなく、\textbf{C-plate のリタデーション}
$\mathrm{Re}_C$(nm)を最適化変数として走査し、式 \eqref{eq:d_from_Re} により
\begin{equation}
  d_C\,[\mu\mathrm{m}] = \frac{\mathrm{Re}_C\,[\mathrm{nm}]}{1000\,\Delta n_C}
\end{equation}
へ換算して膜厚表記を与える。

なお $\mathrm{Re}_C$ の符号は、実装上は ``signed ReC'' として取り扱い、負の場合はTACなどのnegative C-plateを表す。


\subsubsection{相対角(光学軸方位)の定義}

\begin{figure}[htb]
\centering
\includegraphics[width=0.92\linewidth]{azimuth.png}
\caption{xxx}
\label{fig:azimuth}
\end{figure}

\paragraph{基準配置}
本節では,基準配置として以下を採用する。

\begin{itemize}
  \item 入射側偏光板(POL)の吸収軸を $x$ 軸に一致させる:$\mathbf{c}_1=\hat{\mathbf{x}}.$
  \item 液晶(LC)の面内光学軸も $x$ 軸に一致させる: $\mathbf{a}_{\mathrm{LC}}=\hat{\mathbf{x}}.$
  \item 出射側偏光板(アナライザー)の吸収軸を $y$ 軸に一致させる:$\mathbf{c}_2=\hat{\mathbf{y}}.$
\end{itemize}

したがって基準配置では,
$
  \mathbf{c}_1\parallel\hat{\mathbf{x}},\quad
  \mathbf{a}_{\mathrm{LC}}\parallel\hat{\mathbf{x}},\quad
  \mathbf{c}_2\parallel\hat{\mathbf{y}}
$
となり,入射側偏光板と出射側偏光板はクロスニコル条件(透過軸が互いに直交)を満たす。

\paragraph{光学軸の向き}
本プログラムでは汎用性を確保するため、各層の光学軸方位は,基準軸として入力側偏光子(POL)の透過軸を取り,その方位角を $0^\circ$ と定義する.
これに対し,上側 A-plate,下側 A-plate,LC の配向(LC のラビング方向あるいは暗状態の等価主軸),および C-plate の光学軸方位をそれぞれ
\[
\alpha_{\mathrm{A,top}},\ \alpha_{\mathrm{A,bot}},\ \alpha_{\mathrm{LC}},\ \alpha_{\mathrm{C}}
\]
で表し,\textbf{相対角}として
\[
\texttt{relA}=\Delta\alpha_{\mathrm{A,top}}=\alpha_{\mathrm{A,top}}-\alpha_{\mathrm{POL}},\quad
\texttt{lc\_rel}=\Delta\alpha_{\mathrm{LC}}=\alpha_{\mathrm{LC}}-\alpha_{\mathrm{POL}},\quad \text{etc.}
\]
を用いて記述する($\alpha_{\mathrm{POL}}=0^\circ$).
例えば,下側スタック(入力側)の各光軸を,入射偏光子角度 \texttt{pol\_in} に追従して回転する場合には
各層の光軸方位角は次式で定義される:
Case-1 ではスタックを
\[
\mathrm{POL}\ /\ \mathrm{LC}\ /\ \mathrm{A}\ /\ \mathrm{C}\ /\ \mathrm{Analyzer}
\]
とし,面内角度はすべて「偏光板の吸収軸」を基準とする角度として与える。
基準配置として,入射側偏光板(POL)の吸収軸を $x$ 軸($\mathbf{c}_1=\hat{\mathbf{x}}$),
出射側偏光板(Analyzer)の吸収軸を $y$ 軸($\mathbf{c}_2=\hat{\mathbf{y}}$)に一致させる。
このとき,\texttt{pol\_in}, \texttt{pol\_out}(deg)はそれぞれ吸収軸の $z$ 軸回り回転角であり,
正面視において \texttt{pol\_out}=0 はクロスニコル基準(O型偏光板定義に基づく基準配置)を意味することに注意する。
\\
\paragraph{具体例}
Case-1(LC/A/C, absorption-basis)では,LC の面内光学軸は \texttt{pol\_in} に対して相対角 \texttt{relLC} を持ち,
\begin{equation}
  \alpha_{\mathrm{LC}}=\texttt{pol\_in}+\texttt{rel\_LC}
\end{equation}
で与えられる。一方,A-plate は本実装では Analyzer 側(\texttt{pol\_out})を基準として配置され,
A の方位角は
\begin{equation}
  \alpha_{A}=A_{\mathrm{base}}+\texttt{pol\_out}+\texttt{relA}
\end{equation}
で与えられる(Case-1 では $A_{\mathrm{base}}=90^\circ$ に固定)。
したがって Analyzer 側を基準に固定し(\texttt{pol\_out}=0),入力側のみ \texttt{pol\_in} を回転させる場合,
LC のみが \texttt{pol\_in} に追随し,A は固定角のままとなり,独立に設定することが可能となる。

実際の計算では
(\texttt{pol\_in}=$0^\circ$, \texttt{pol\_out}=$0^\circ$),
\texttt{relA}=$0.25^\circ$, \texttt{rel\_LC}=$0.25^\circ$ とした場合,
\begin{equation}
\alpha_{\mathrm{pol,in}}^{(\mathrm{abs})}=0^\circ,\qquad
\alpha_{\mathrm{LC}}=0.25^\circ,\qquad
\alpha_{\mathrm{A}}=90.25^\circ,\qquad
\alpha_{\mathrm{pol,out}}^{(\mathrm{abs})}=90^\circ.
\end{equation}
とした。
\\

\begin{table}[t]
  \centering
  \caption{本節で用いる材料定数と固定パラメータ(代表値)。}
  \label{tab:model_params_case12}
  \begin{tabular}{lcc}
    \toprule
    Parameter & Symbol & Value \\
    \midrule
    LC birefringence & $\Delta n_{\mathrm{LC}}$ & $0.10$ \\
    LC thickness & $d_{\mathrm{LC}}$ & $3.1~\mu\mathrm{m}$ \\
    A-plate birefringence (upper) & $\Delta n_{A,\mathrm{up}}$ & $0.00142$ \\
    A-plate birefringence (lower) & $\Delta n_{A,\mathrm{low}}$ & $0.00145$ \\
    C-plate birefringence & $\Delta n_{C}$ & $0.12049$ \\
    LC retardation (design) & $\mathrm{Re}_{\mathrm{LC}}$ & $310~\mathrm{nm}$ \\
    A-plate base retardation (each) & $\mathrm{Re}_{A,\mathrm{base}}$ & $155~\mathrm{nm}$ \\
    \bottomrule
  \end{tabular}
\end{table}


\begin{table}[t]
  \centering
  \caption{本節で用いる材料定数と固定パラメータ(代表値)および角度パラメータ(定義は本文参照)。}
  \label{tab:model_params_case12}
  \begin{tabular}{lcc}
    \toprule
    Parameter & Symbol & Value \\
    \midrule
    \multicolumn{3}{l}{\textbf{(A) Material / Retardation parameters}}\\
    LC birefringence & $\Delta n_{\mathrm{LC}}$ & $0.10$ \\
    LC thickness & $d_{\mathrm{LC}}$ & $3.1~\mu\mathrm{m}$ \\
    A-plate birefringence (upper) & $\Delta n_{A,\mathrm{up}}$ & $0.00142$ \\
    A-plate birefringence (lower) & $\Delta n_{A,\mathrm{low}}$ & $0.00145$ \\
    C-plate birefringence & $\Delta n_{C}$ & $0.12049$ \\
    LC retardation (design) & $\mathrm{Re}_{\mathrm{LC}}$ & $310~\mathrm{nm}$ \\
    A-plate base retardation (each) & $\mathrm{Re}_{A,\mathrm{base}}$ & $155~\mathrm{nm}$ \\
    \midrule
    \multicolumn{3}{l}{\textbf{(B) Angle parameters (in-plane, deg)}}\\
    Input POL absorption-axis rotation & \texttt{pol\_in}  & $0.5^\circ$ \\
    Output POL absorption-axis rotation & \texttt{pol\_out} & $0.0^\circ$ \\
    Lower A relative rotation vs. input POL & \texttt{relA} $\equiv$ \texttt{rel\_rot\_LA} & $0.5^\circ$ \\
    Upper A relative rotation vs. output POL & \texttt{rel\_rot\_UA} & $0.0^\circ$ \\
    LC relative rotation vs. input POL & \texttt{lc\_rel} $\equiv$ \texttt{LC\_REL\_TO\_INPOL} & $1.0^\circ$ \\
    \midrule
    \multicolumn{3}{l}{\textbf{(C) Derived azimuths used in the stack (deg)}}\\
    Input-side azimuth (POL) & $\alpha_{\mathrm{pol,in}}$ & $\texttt{pol\_in}$ \\
    Lower A azimuth & $\alpha_{A,\mathrm{bot}}$ & $\texttt{pol\_in}+\texttt{relA}$ \\
    LC azimuth & $\alpha_{LC}$ & $\texttt{pol\_in}+\texttt{lc\_rel}$ \\
    Upper A azimuth & $\alpha_{A,\mathrm{top}}$ & $90^\circ+\texttt{pol\_out}+\texttt{relA}$ \\
    Output-side azimuth (POL) & $\alpha_{\mathrm{pol,out}}$ & $90^\circ+\texttt{pol\_out}$ \\
    \bottomrule
  \end{tabular}
\end{table}

\subsection{最適化および出力}
最適化は、所望視角点 $(\theta,\phi)=(30^\circ,45^\circ)$ の漏れ最小化(CR 最大化)を主目的とし、
同時に視野角領域(本実装では $\theta \le 60^\circ$ の格子)における暗状態の崩れ具合を
CR 等高線(ISO)として可視化する。
最適化および可視化は \texttt{ips\_compensation4.py} により実行し、以下の成果物を生成する:
(i) \texttt{summary.csv}(各ケースのベースラインと最良解の指標集約)、
(ii) \texttt{best\_stack\_caseX.json}(最良スタックの層パラメータ)、
(iii) \texttt{progress\_caseX.csv}(探索の進捗:更新番号・最良 CR・Stokes 追跡値)、
(iv) \texttt{stokes\_white\_caseX.csv}(白色合成での Stokes 追跡)、
(v) \texttt{iso\_X\_best.png}(最良解の CR 等高線図)などである。

\subsubsection{評価指標:CR 等高線図、$\mathrm{CR}_{00}$、および Stokes 追跡}
暗状態漏れは透過率 $T_{\mathrm{leak}}$ から $\mathrm{CR}=1/T_{\mathrm{leak}}$(実装の定義に従う)として算出し、
視野角格子上の CR を等高線(ISO)として可視化する。
特に、正面視($\theta=0^\circ$)に対応する $\mathrm{CR}_{00}$ は、暗状態中心の性能を代表する指標として併記する。

また、最良解の理解のため、スタック途中(例:\texttt{POL\_in}、\texttt{el\#0}、\texttt{el\#1}、\texttt{el\#2})
における偏光状態を Stokes パラメータ $(s_1,s_2,s_3)$ として追跡する。
ここで $s_3$ は円偏光成分(楕円率)に相当し、$|s_3|\ll 1$ はほぼ線偏光を意味する。
さらに、解析用に \texttt{case1\_results\_20260108\_152708.txt} では、
視角 $(\theta,\phi)$ における解析器吸収軸の方位角 $\alpha$(transverse basis 上)と
最終 Stokes から
\begin{equation}
  I(\alpha)=\frac{1}{2}\left(1+s_1\cos 2\alpha + s_2\sin 2\alpha\right),
  \qquad
  T_{\mathrm{leak,pred}}=\frac{1}{2}I(\alpha)
  \label{eq:Tleak_pred_from_stokes}
\end{equation}
により漏れを推定し、数値計算の $T_{\mathrm{leak}}$ と整合することを確認している
(本節でも同様の観点で Stokes を解釈する)。

