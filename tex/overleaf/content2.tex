\section{例題(プログラム実行例)}
本節では、\texttt{ips\_compensation.py} が提供する代表的な解析手順を、三つの例題として整理する。
いずれも視野角依存の暗状態漏れを $\mathrm{CR}$ の等高線(ISO)として可視化し、加えて設計指標($\mathrm{CR}_0$、$\mathrm{frac}\{\mathrm{CR}>\mathrm{thr}\}$、distortion 指標)を数値として併記する。
なお、ここで示す ``マッチング(matched)'' とは、LC と補償板(主に A-plate)の複屈折 $\Delta n(\lambda)$ の相対的な波長依存性が整合している状態、すなわち B/G/R 等の代表波長に対して $\Delta n$ のスケーリング係数が同一(または同等)であり、位相遅れ $\Gamma(\lambda)\propto \Delta n(\lambda)\,d/\lambda$ の波長依存が相互に補償し合う状態を指す。
一方 ``ミスマッチ(mismatched)'' では、LC と A-plate の $\Delta n(\lambda)$ スケーリングが異なるため、設計波長(典型的には G)で最適化された補償条件が他波長で成立せず、白色評価(B/G/R の重み付き平均)における $\mathrm{CR}_0$ が低下し得る。

\subsection{用語の定義:phase dispersion と mismatch}
本仕様書では、位相遅れ(retardation)を
\[
\Gamma(\theta,\phi,\lambda)=\frac{2\pi}{\lambda}\,\Delta n_{\mathrm{eff}}(\theta,\phi,\lambda)\,d
\]
と表し、その波長依存性を \textbf{phase dispersion} と呼ぶ。
ここで $\Delta n_{\mathrm{eff}}$ は入射角に依存する実効複屈折である。
\textbf{dispersion mismatch} とは、異なる要素(LC と A-plate 等)で $\Delta n(\lambda)$ の相対スケールが異なることにより、$\Gamma(\lambda)$ の波長依存が相殺されず、暗状態の干渉条件(偏光状態の打ち消し)が波長間で一致しない状況を指す。

\subsubsection{R_0, R_{\mathrm{th}} の定義(A-plate / C-plate / LC との関係を含む)}

\paragraph{一般的な定義}
フィルムの主屈折率を $(n_x,n_y,n_z)$、膜厚を $d$ とすると、一般に
\begin{align}
R_0 &\equiv (n_x-n_y)\,d 
\quad \text{(面内リタデーション; normal入射での面内位相遅れ長)}\\
R_{\mathrm{th}} &\equiv \left(n_z-\frac{n_x+n_y}{2}\right)d
\quad \text{(厚み方向リタデーション; biaxial filmの代表的定義)}
\end{align}
と定義されます(規格・流儀により $R_{\mathrm{th}}=(n_z-n_y)d$ 等の別定義もありますが、本資料では上式を採用します)。

位相遅れ(リターダの遅相量)$\Gamma$ は一般に
\begin{align}
\Gamma \;=\; \frac{2\pi}{\lambda}\,R
\end{align}
の形で現れ、$\lambda$ は真空波長、$R$ はリタデーション長(m)です。
本コードでも $\Gamma$ は $\frac{2\pi}{\lambda}$ を前に持つ形で実装されています。:contentReference[oaicite:0]{index=0}

\paragraph{A-plate / C-plate / LC との関係(概念)}
IPS補償の文脈では、$R_0$ と $R_{\mathrm{th}}$ は各補償板の役割に直結します。
\begin{itemize}
  \item \textbf{A-plate(面内補償):} 主に \textbf{$R_0$}(面内異方性)を持つ層として機能し、
        画素配向や入射方位 $\phi$ に対する位相補償($\phi$依存の補償)を担います。
  \item \textbf{C-plate(厚み方向補償):} 主に \textbf{$R_{\mathrm{th}}$}(厚み方向の異方性)を持つ層として機能し、
        斜入射 $\theta$ によって顕在化する位相ずれを補償します($\phi$依存が相対的に弱い成分を担う)。
  \item \textbf{LC(液晶層):} 配向角を持つ異方層として $R_0$ 相当の位相遅れを生成し、
        これが暗状態漏れ(Tleak)や視野角依存性の主因となります。
        A/C 補償板はこの LC 起因の位相遅れを相殺するように設計されます。
\end{itemize}

\subsection{例題2:$A$ スケール×$C$ 膜厚グリッド掃引と $\mathrm{frac}\{\mathrm{CR}>\mathrm{thr}\}$}

\subsubsection{目的}
\texttt{run\_Ascale\_C\_grid\_check()} は、$A\_\mathrm{scale}$ と C-plate 膜厚(符号付き)の 2D グリッドに対して ISO 図を生成し、さらに固体角重み付きで
\[
\mathrm{frac}\{\mathrm{CR}>\mathrm{thr}\}
\]
(閾値 $\mathrm{thr}$ を満たす視野の割合)を算出する。
\subsubsection{構成}

概ねあなたの理解(POL/TAC/…/POL)で合っていますが、\textbf{注意点が2つ}あります。

\begin{enumerate}
  \item \textbf{コード上の \texttt{stack} には POL は入っていません}(偏光子は層ではなく、\texttt{pol\_axes()} が返す吸収軸 \texttt{c1,c2} として Tleak 計算に入ります)。
  \item \textbf{C-plate は ``入側C'' と ``出側C'' の2枚構成になり得ます}(ただし現コードは \emph{正のC} のときだけ入側Cが入り、signedC対応の出側Cは符号付きで常に入る、という実装になっています)。
\end{enumerate}

\paragraph{1) run\_ex2 の層構成は?}
\texttt{ips\_compensation\_run\_signedC.py} の \texttt{build\_stack\_realistic()} が作る retarder stack は次です(POLは別扱い)。

\paragraph{retarder stack(コード上の順序)}
\begin{itemize}
  \item \textbf{TAC(in)} $\times$ \texttt{tac\_repeat}
  \item \textbf{C(in)}(※ \texttt{dC\_um > 0} のときのみ入る)
  \item \textbf{A(lower)}
  \item \textbf{LC}
  \item \textbf{A(upper)}
  \item \textbf{C(out)}(※ \texttt{abs(dC\_um)>0} なら入る。\texttt{dC\_um<0} だと dn の符号反転)
  \item \textbf{TAC(out)} $\times$ \texttt{tac\_repeat}
\end{itemize}

したがって、概念的に``全部込み''で書くなら:
\begin{center}
\textbf{POL(in)} / TAC / C / A / LC / A / C / TAC / \textbf{POL(out)}
\end{center}
ただし \texttt{dC\_um=0} の場合は Cが無いので:
\begin{center}
\textbf{POL} / TAC / A / LC / A / TAC / \textbf{POL}
\end{center}
また \texttt{dC\_um<0} の場合は(現実装では)入側Cが入らず、出側Cのみが「負C」として入ります。

\paragraph{2) 各層のリタデーション・膜厚・屈折率はどう決まる?}
\texttt{ips\_compensation\_run\_signedC.py} の定数・式に完全に従います。

\subparagraph{共通:基準屈折率}
\begin{itemize}
  \item すべての層で \textbf{\texttt{no = NO\_BASE = 1.50}}
  \item \textbf{\texttt{ne = no + dn}}(G波長を基準とした値を \texttt{stack} に格納)
\end{itemize}

\subparagraph{LC}
定数(ファイル内):
\begin{itemize}
  \item \texttt{dn\_LC = 0.100}
  \item \texttt{d\_LC = 3.1e-6} [m]
\end{itemize}
よって \textbf{LCのG基準の面内リタデーション}は
\[
Re_{LC,G} = dn_{LC}\, d_{LC} = 0.1\times 3.1\ \mu\mathrm{m} = 310\ \mathrm{nm}
\]
(この 310 nm が Rebase の起点です)

\subparagraph{A-plate(上下2枚)}
定数:
\begin{itemize}
  \item \texttt{RE\_LC\_NM = 310.0}
  \item \texttt{RE\_A\_EACH\_BASE\_NM = RE\_LC\_NM/2 = 155.0} [nm]
  \item \texttt{dn\_lowerA = 0.00145}
  \item \texttt{dn\_upperA = 0.00142}
\end{itemize}
まず各A板の \textbf{G基準リタデーション}を
\[
Re_{A,each,G} = 155\ \mathrm{nm}\times A_{\mathrm{scale}}
\]
と置き、膜厚を
\[
d_{\mathrm{lower}} = \frac{Re_{A,each,G}}{dn_{\mathrm{lowerA}}},\quad
d_{\mathrm{upper}} = \frac{Re_{A,each,G}}{dn_{\mathrm{upperA}}}
\]
で決めています(コードは m 単位に直して計算)。

したがって \textbf{\texttt{A\_scale} は「A板のRe(≒膜厚)を一括倍率するノブ」}です。

\subparagraph{C-plate(axis=[0,0,1] の単軸)}
定数:
\begin{itemize}
  \item \texttt{dn\_C = 0.12049}
  \item 膜厚:\texttt{d = abs(dC\_um)*1e-6} [m]
\end{itemize}

\begin{itemize}
  \item \texttt{dC\_um > 0} のとき:入側Cが \textbf{dn=+dn\_C} で追加される(現コードの ``L-C optional'')
  \item 出側Cは signed 対応で、常に
    \begin{itemize}
      \item \texttt{dnC\_eff = dn\_C * sign(dC\_um)}(符号反転あり)
      \item \texttt{d = abs(dC\_um)}(厚みは絶対値)
    \end{itemize}
\end{itemize}
従って \textbf{負Cは「厚み負」ではなく「dn符号反転」}として実装されています。

\subparagraph{TAC(axis=[0,0,1] の単軸)}
run\_ex2 では定数配列の先頭が使われ、
\begin{itemize}
  \item \texttt{tac\_repeat = TAC\_REPEATS[0] = 1}
  \item \texttt{tac\_um = TAC\_UM\_CASES[0] = 40.0} [\,$\mu$m\,]
  \item \texttt{dn\_tac = DN\_TAC\_CASES[0] = -0.0020}
\end{itemize}
膜厚:
\[
d_{\mathrm{TAC}} = 40\ \mu\mathrm{m}
\]

※実装上、TACは \texttt{tac\_layer = \{"type":"C", ...\}} として \texttt{type} が \texttt{"C"} になっています(概念上はTACですが、\textbf{現在 DN\_SCALE で C と TAC がどちらも 1.0 なので結果は同じ}です。将来TAC分散を入れたいなら \texttt{type:"TAC"} に直すのが筋です)。

\paragraph{3) 分散(matched / mismatched)はどう入っている?}
\texttt{stack} に入っている \texttt{no, ne} は \textbf{G波長の値}として扱われます。
評価時に \texttt{\_ne\_no\_for\_wl(el, wl\_key)} が呼ばれて、
\begin{itemize}
  \item まず \texttt{dnG = neG - no}
  \item \texttt{USE\_DN\_DISPERSION=False} なら \textbf{dnは波長で変えない}
  \item \texttt{USE\_DN\_DISPERSION=True} なら、層typeごとに
  \[
  dn(\lambda) = dn_G \times \text{DN\_SCALE[type][wl\_key]}
  \]
  として \textbf{dnだけをスケール}します。
\end{itemize}

\texttt{DN\_SCALE} は例えば:
\begin{itemize}
  \item matched:LCとAが同じ (B:1.10, G:1.00, R:0.90)
  \item mismatched:LCは同じだが、Aだけ (B:1.20, G:1.00, R:0.80)
\end{itemize}
という「dnの色分散の相対ズレ」を与えます。

※なお、分散OFFでも位相遅れ $\Gamma$ には \textbf{$1/\lambda$} が入るので、波長でCRが変わる要因は残ります(dn自体は固定でも、$\Gamma\propto dn\cdot d/\lambda$ の $\lambda$ が変わる)。

\paragraph{4) 相対角度(POL / A / LC)はどう決まる?}
\subparagraph{偏光子(POL)}
\texttt{pol\_axes(pol\_in, pol\_out)} で
\begin{itemize}
  \item 入側POL軸 \texttt{c1}:x軸を \texttt{pol\_in} だけ回転
  \item 出側POL軸 \texttt{c2}:y軸を \texttt{pol\_out} だけ回転(crossed系)
\end{itemize}

\subparagraph{A板(LA/UA)}
\begin{itemize}
  \item LA の基準方位:0$^\circ$
  \item UA の基準方位:90$^\circ$
  \item LA軸:\texttt{pol\_in + rel\_rot\_LA\_deg}
  \item UA軸:\texttt{pol\_out + rel\_rot\_UA\_deg}
\end{itemize}
run\_ex2 は \texttt{relA=0.5} を渡して \textbf{LA/UAとも +0.5$^\circ$} にしているので、A板が``効く''ようにわざと微小ミスアライメントを作っています。

\subparagraph{LC}
LC軸は
\[
\texttt{pol\_in + lc\_rel\_to\_inpol\_deg}
\]
(run\_ex2 だと \texttt{lc\_rel=1.0} なので pol\_in から +1$^\circ$)です。

\subsubsection{Rebase(\texttt{RE\_A\_EACH\_BASE\_NM})はどう決めているか}

このコードでは、Rebaseは\textbf{定数として決め打ち}しています。
\texttt{ips\_compensation\_run\_signedC.py} の冒頭で次のように定義されています。

\begin{verbatim}
RE_LC_NM = 310.0
RE_A_EACH_BASE_NM = RE_LC_NM / 2.0
\end{verbatim}

つまり、
\begin{itemize}
  \item LCの基準リタデーション($Re_{\mathrm{LC}}$)を $310\,\mathrm{nm}$ と仮定する。
  \item A-plateが上下2枚ある前提で、各A板の基準リタデーションをその半分に配分する。
\end{itemize}

よって各A板の基準リタデーションは
\[
Re_{\mathrm{A,each,base}} = 155\,\mathrm{nm}
\]
となります。

実際に使うA板のリタデーションは、スケール係数 \texttt{A\_scale} により
\[
Re_{\mathrm{A,each}} = Re_{\mathrm{A,each,base}} \times A_{\mathrm{scale}}
\]
としています。

さらに膜厚は(その波長での)複屈折 $\Delta n$ で割って決めています:
\[
d = \frac{Re_{\mathrm{A,each}}}{\Delta n}
\]

要するに、Rebaseは「LCの基準Reを $310\,\mathrm{nm}$ とし、その半分を各A板の基準にする」という\textbf{設計仮定}です。
(mismatchedでもGが同じ値になりやすいのは、スケールがG基準で正規化されているためです。)

\begin{table}[t]
\centering
\caption{Example22: $CR(\theta=0,\phi=0)$ for matched / mismatched (RGB and White)}
\label{tab:ex22_cr00}
\begin{tabular}{c|rrrr}
\hline
\multicolumn{5}{c}{\textbf{case = matched}} \\
\hline
$A_{\mathrm{scale}}$ & $CR00_{\mathrm{B}}$ & $CR00_{\mathrm{G}}$ & $CR00_{\mathrm{R}}$ & $CR00_{\mathrm{W}}$ \\
\hline
0.50 &  4574.43 &  2025.91 &  1837.45 &  2387.69 \\
1.00 & 16679.71 &  4008.35 &  2785.22 &  4487.86 \\
1.50 & 30168.36 & 19594.48 &  6490.64 & 12591.81 \\
2.00 &  5817.15 & 21748.35 & 32611.60 & 12070.01 \\
\hline
\multicolumn{5}{c}{\textbf{case = mismatched}} \\
\hline
$A_{\mathrm{scale}}$ & $CR00_{\mathrm{B}}$ & $CR00_{\mathrm{G}}$ & $CR00_{\mathrm{R}}$ & $CR00_{\mathrm{W}}$ \\
\hline
0.50 &  4907.44 &  2025.91 &  1788.24 &  2387.42 \\
1.00 & 24419.69 &  4008.35 &  2465.20 &  4309.87 \\
1.50 & 16875.28 & 19594.48 &  4588.28 &  9139.66 \\
2.00 &  4311.89 & 21748.35 & 14266.28 &  8620.87 \\
\hline
\end{tabular}
\end{table}

\subsubsection{結果:A-plate, C-plateの膜厚依存性}
A-plate と C-plate はそれぞれ異なる角度依存の位相遅れを与えるため、膜厚は ISO-plot の等高線形状を直接変形させる。
一般に、A-plate は主として面内の位相補償、C-plate は入射角に対する等方的($\phi$ 依存の弱い)補償成分を担う。
そのため、$C$ 膜厚を掃引すると、低 $\theta$ での補償最適点だけでなく、斜視での暗状態漏れが支配的となる領域(高 $\theta$)の等高線が系統的に移動する。
$\mathrm{frac}\{\mathrm{CR}>\mathrm{thr}\}$ は ISO-plot を単一スカラーに要約する指標であり、設計探索のランキングに有効である。

\subsubsection{結果:波長分散依存性}
\texttt{demo\_CR0\_dispersion\_effect()} は、LC と A-plate の分散整合(matched)と不整合(mismatched)を切り替え、白色評価の $\mathrm{CR}_0$ が低下し得ることを数値的に示す。
設定は以下に固定する:
(i) 偏光子と A-plate はペアとして回転(pair rotation)、
(ii) ペア内部の A 軸ズレを $\mathrm{REL\_ROT}\_*=\SI{0.5}{\degree}$ として A-plate の寄与を顕在化、
(iii) LC director の入力側偏光子に対するズレを $\SI{1.0}{\degree}$ とする。
\\
G 波長で最適化した $A\_\mathrm{scale}$ は、$\Gamma(\lambda)$ の条件が \emph{G のみ}で満たされるように決まる。
matched では LC と A の $\Delta n(\lambda)$ が同様にスケールするため、B/R でも位相遅れ条件が近似的に維持され、白色平均の $T_{\mathrm{leak}}$ が抑制される。
一方 mismatched では、A の $\Delta n(\lambda)$ が LC より急峻(または緩やか)に変化するため、B/R で $\Gamma$ の過補償/不足補償が発生し、暗状態の偏光状態が完全に打ち消されない。
この残差が波長平均後の $T_{\mathrm{leak}}$ を増大させ、結果として
\[
\mathrm{CR}_0 = \frac{T_{\mathrm{bright}}}{\langle T_{\mathrm{leak}}\rangle_{\lambda}}
\]
が低下する。
\\
\textbf{$\mathrm{CR}_0$ 出力}は $\theta=0$ における $\phi$ 平均(および最小・最大)として表示される。
理想的には $\theta=0$ で $\phi$ 依存は弱いが、数値安全性と一般化のため平均化処理を保持している。
\textbf{ISO-plot} は $\theta$--$\phi$ 上の $\mathrm{CR}$ 等高線であり、暗状態漏れが視野角でどのように増大するか(補償の破綻角度・方位)を定性的かつ定量的に把握できる。

\subsection{例題3:片側 POL+A ペアの回転ズレによる ISO-plot の distortion}
\subsubsection{対象関数}
\texttt{run\_pol\_in\_distortion\_check()} は、$A\_\mathrm{scale}$ と $C$ 膜厚を固定したまま入力側ペア回転 $\mathrm{pol\_in}$ を掃引し、ISO-plot の ``歪み(distortion)'' を可視化する。
併せて簡易な方位異方性指標(phi-anisotropy metric)を算出し、等高線が方位方向に引き延ばされる傾向を確認する。

\subsubsection{物理的解釈}
片側のみのペア回転は、LC director と入射側偏光子の相対角度を変化させると同時に、下側 A-plate の主軸配向も変化させる。
このとき、上側(解析側)スタックが固定であるため、視野角依存の偏光状態の回転と位相遅れが方位角 $\phi$ に対して非対称となり、暗状態漏れが特定方位で顕著に増大する。
結果として ISO-plot の等高線は等方的な縮退形状から外れ、方位方向に歪んだ形状(distortion)を示す。

\subsubsection{Example 3(pol\_in distortion check):main から計算・保存までの呼び出しフロー(詳細)}

Example 3(\texttt{run\_ex3\_fixed.py})は、入力偏光子の回転角 \texttt{pol\_in} を複数点スイープし、
各 \texttt{pol\_in} について視野角全体のコントラスト比 $CR(\theta,\phi)$ を計算して、
(1) 閾値以上の視野割合 \texttt{frac(CR>thr)} と、
(2) 方位角方向($\phi$方向)のばらつき指標 \texttt{phi\_std(log10CR)}
を求め、ISO-contrast 図とともに保存する手順である。

\paragraph{(1) \texttt{main()}:角度グリッドと描画設定}
\texttt{main()} 冒頭で、ISO図(極座標プロット)および指標計算に用いる角度グリッドを設定する。
\begin{itemize}
  \item \texttt{THETA\_MAX = 60.0}:入射角 $\theta$ の最大値(度)
  \item \texttt{DTHETA = 5.0}, \texttt{DPHI = 5.0}:それぞれ $\theta$ と $\phi$ の刻み(度)
  \item \texttt{theta\_ticks = [0,10,20,30,40,50,60]}:極座標プロットの同心円ラベル($\theta$ を度表記)
\end{itemize}
ここで \texttt{theta\_ticks} を明示することで、Matplotlib のデフォルト動作による
$r$軸(ラジアン)の自動目盛(例:0.2, 0.4, ...)の表示を避け、同心円を $\theta$ [deg] として表示できる。

\paragraph{(2) signed-C の扱い:\texttt{C\_um=0.0} の意味}
Example 3 では \texttt{C\_um = 0.0} を設定し、C-plate を挿入しない条件を再現する。
\begin{itemize}
  \item signedC 版では \texttt{C\_um < 0} は「負の C-plate(Cの複屈折符号反転)」を意味する。
  \item 旧仕様で用いられていた「負値=C板なし」を混同しないため、\texttt{C\_um=0.0} を明示して
        「C板なし」を表す。
\end{itemize}

\paragraph{(3) 本体:\texttt{run\_pol\_in\_distortion\_check(...)} の呼び出し}
\texttt{main()} は次に、スイープ本体である
\texttt{run\_pol\_in\_distortion\_check(...)} を呼び出す。
この関数は \texttt{pol\_in\_list} の各値に対して、スタック構築→CRグリッド計算→指標算出→図保存を行い、
最後に結果一覧を \texttt{manifest.json} としてまとめて保存する。

\paragraph{(4) \texttt{for pol\_in in pol\_in\_list:}:各 pol\_in の処理}
スイープのループでは、各 \texttt{pol\_in}(入力偏光子の回転角)について以下を行う。

\subparagraph{(4-1) 偏光軸の生成:\texttt{pol\_axes(pol\_in, pol\_out)}}
まず、入力偏光子(\texttt{pol\_in})と出力偏光子(\texttt{pol\_out})の設定から、
解析に用いる偏光軸(ベクトル)を生成する。
\begin{itemize}
  \item \texttt{c1, c2 = pol\_axes(pol\_in, pol\_out)}
\end{itemize}
この \texttt{c1,c2} は、後段の透過計算(Tleak 計算)で「偏光子が通す成分」を定義する基底として使われる。

\subparagraph{(4-2) スタック生成:\texttt{build\_stack\_realistic(...)}}
次に、補償板(A-plate 等)と LC を含む光学スタックを生成する。
Example 3 では通常、
\begin{itemize}
  \item \texttt{A\_scale = 2.0}(A-plate の Re を倍率スケールするノブ)
  \item \texttt{dC\_um = C\_um}(ここでは 0.0 で C板なし)
\end{itemize}
を用いてスタックを構築する。
\[
\texttt{stack} \leftarrow \texttt{build\_stack\_realistic}(dC\_um=C\_um,\ A\_scale=2.0,\ \ldots)
\]
この \texttt{stack} は各層の(波長依存を含む)屈折率・複屈折・膜厚・配向角などを保持し、
後段の Tleak / CR 計算の入力となる。

\subparagraph{(4-3) CRグリッド計算:\texttt{compute\_CR\_grid(...)}}
スタックと偏光軸が決まったら、視野角全体でのコントラスト比 $CR(\theta,\phi)$ を計算する。
\begin{itemize}
  \item \texttt{thetas, phis, CR = compute\_CR\_grid(stack, c1, c2, theta\_max=THETA\_MAX, dtheta=DTHETA, dphi=DPHI)}
\end{itemize}
\texttt{compute\_CR\_grid} は $\theta$ と $\phi$ の2次元グリッド上で、
各点の漏れ光透過率 \texttt{Tleak} を計算し、\texttt{CR\_from\_Tleak} により CR に変換して配列として返す。
(White評価の場合は B/G/R を内部で平均して 1つの \texttt{Tleak} として扱う。)

\subparagraph{(4-4) 指標計算}
得られた $CR(\theta,\phi)$ 配列から、設計評価用のスカラー指標を計算する。

\begin{itemize}
  \item \textbf{\texttt{frac(CR>thr)}}:
    閾値 \texttt{thr} を満たす視野の割合を、固体角重み付きで算出する。
    すなわち、各グリッド点の立体角要素(概ね $\sin\theta$ を含む重み)で重み付けした上で、
    $CR(\theta,\phi)>\texttt{thr}$ の領域の比率を求める。
  \item \textbf{\texttt{phi\_std(log10CR)}}:
    $\log_{10}(CR)$ を取り、各 $\theta$ における $\phi$ 方向のばらつきを標準偏差として要約し、
    さらにそれを単一指標としてまとめる(\texttt{phi}方向の歪み・ムラ感を表す簡易指標)。
\end{itemize}

\subparagraph{(4-5) ISO-contrast 図の保存:\texttt{plot\_isocontrast\_polar(...)}}
次に、計算した $CR(\theta,\phi)$ を極座標等高線図として可視化し、PNGに保存する。
\begin{itemize}
  \item \texttt{plot\_isocontrast\_polar(thetas, phis, CR, ..., theta\_ticks=theta\_ticks)}
\end{itemize}
\texttt{theta\_ticks} を渡すことで、同心円(半径方向)は $\theta$ [deg] の指定値で表示される。

\paragraph{(5) \texttt{manifest.json} の保存}
全 \texttt{pol\_in} の処理が終わると、各 \texttt{pol\_in} に対して
\texttt{frac(CR>thr)}、\texttt{phi\_std(log10CR)}、および保存したプロットファイル名などを
リストとしてまとめ、\texttt{manifest.json} として出力ディレクトリに保存する。
これにより、後からスイープ結果を機械的に集計・再プロットできる。


\clearpage
\section{参考文献}
\begin{thebibliography}{9}
\bibitem{Lee2006AO}
J.-H. Lee, H. Choi, S. H. Lee, J.-C. Kim, and G.-D. Lee,
``Optical configuration of a horizontal-switching liquid-crystal cell for improvement of the viewing angle,''
\emph{Applied Optics} \textbf{45}(28), 7279--7285 (2006).
\end{thebibliography}
