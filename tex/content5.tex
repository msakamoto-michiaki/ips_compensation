\subsection{例題2-2:波長分散のマッチング(matched/mismatched)が正面コントラスト $\mathrm{CR}_{00}$ に与える影響(\texttt{ex22.py})}
本例題では,補償板と LC の複屈折分散 $\Delta n(\lambda)$ の\textbf{相対的な波長依存性}が整合している場合(matched)と,整合していない場合(mismatched)とで,暗状態の正面漏れ $T_{\mathrm{leak}}(0,0)$,および $\mathrm{CR}_{00}=1/T_{\mathrm{leak}}(0,0)$ がどの程度変化するかを定量化する.
ここでは C-plate 条件を $C_{\mu\mathrm{m}}=+1.0$ に固定し,A-plate のスケール因子 $A_{\mathrm{scale}}$ を離散値($0.5,\,1.0,\,1.5,\,2.0$)で走査した.各条件について B/G/R および白色(B/G/R 平均)の $\mathrm{CR}_{00}$ を算出し,結果を \texttt{CR00\_table\_matched\_mismatched.csv} として出力した(\texttt{examples/ex22\_CR00\_table/}).

\subsubsection{結果の要約(数値比較)}
表\ref{tab:ex22_cr00}に $\mathrm{CR}_{00}$(白色)を示す.matched では $A_{\mathrm{scale}}\approx 1.5$ で $\mathrm{CR}_{00,\mathrm{W}}$ が最大となり($2.87\times 10^4$),$A_{\mathrm{scale}}=2.0$ でも同程度の高い正面コントラスト($2.80\times 10^4$)を維持する.一方 mismatched では,$A_{\mathrm{scale}}$ が大きい領域ほど $\mathrm{CR}_{00,\mathrm{W}}$ が低下し,特に正面最適近傍での劣化が顕著となった($A_{\mathrm{scale}}=1.5$ で $2.36\times 10^4$,$A_{\mathrm{scale}}=2.0$ で $2.27\times 10^4$).
matched に対する相対比でみると,$A_{\mathrm{scale}}=1.5$ で約 $-18\%$,$A_{\mathrm{scale}}=2.0$ で約 $-19\%$ の低下に相当する.

\begin{table}[t]
  \centering
  \caption{例題2-2:matched/mismatched における白色 $\mathrm{CR}_{00}$ の比較($C_{\mu\mathrm{m}}=+1.0$ 固定).}
  \label{tab:ex22_cr00}
  \begin{tabular}{c|cc|c}
    \hline
    $A_{\mathrm{scale}}$ &
    $\mathrm{CR}_{00,\mathrm{W}}$ (matched) &
    $\mathrm{CR}_{00,\mathrm{W}}$ (mismatched) &
    mismatched / matched \\
    \hline
    0.5 & $8.35\times 10^3$ & $8.35\times 10^3$ & 1.00 \\
    1.0 & $1.41\times 10^4$ & $1.37\times 10^4$ & 0.97 \\
    1.5 & $2.87\times 10^4$ & $2.36\times 10^4$ & 0.82 \\
    2.0 & $2.80\times 10^4$ & $2.27\times 10^4$ & 0.81 \\
    \hline
  \end{tabular}
\end{table}

\subsubsection{波長別の寄与と「分散ミスマッチ」の意味}
本実装の mismatched 設定では,G チャネルの $\mathrm{CR}_{00,\mathrm{G}}$ が matched と完全に一致した一方で,B/R が条件によって増減し,結果として白色 $\mathrm{CR}_{00,\mathrm{W}}$ が変化した.これは,\textbf{分散の整合を「基準波長(ここでは G)で合わせた上で,B/R における位相遅れの相対誤差を導入する」}という定義に対応する.
すなわち,mismatched では G では設計通りに補償が成立しても,B/R では補償残差(楕円偏光成分)が残り,アナライザ透過後の正面漏れ $T_{\mathrm{leak}}(0,0)$ が増加し得る.
実際,$A_{\mathrm{scale}}$ が大きい領域では,R チャネルの漏れが増えることで $\mathrm{CR}_{00,\mathrm{W}}$ が低下する傾向が見られた(例:$A_{\mathrm{scale}}=2.0$ では $\mathrm{CR}_{00,\mathrm{R}}$ が $4.41\times 10^4$ から $3.07\times 10^4$ に低下).

\subsubsection{設計上の含意}
以上より,\textbf{分散マッチングは「正面コントラストのピーク値」だけでなく,「ピーク近傍での安定性(白色での劣化抑制)」に効く}ことが分かる.
特に,A-plate の retardation を強めて高い $\mathrm{CR}_{00}$ を狙う設計($A_{\mathrm{scale}}\gtrsim 1.5$)では,B/R での位相遅れ誤差が白色漏れを支配しやすく,mismatched 条件での $\mathrm{CR}_{00,\mathrm{W}}$ 劣化が顕在化する.
したがって,白色暗状態の正面コントラストを高く保つには,
(1) LC と補償板の $\Delta n(\lambda)$ の相対分散を整合させる(matched 化),
(2) あるいは分散ミスマッチを見込んで A-plate 設計点を保守側に寄せる/追加補償自由度を導入する,
といった方針が有効である.

% % ---- 図(別途貼り付け:波長別 CR00 の棒グラフや,matched/mismatched の比較図など) ----
% \begin{figure}[t]
%   \centering
%   % \includegraphics[width=0.8\linewidth]{ex22_CR00_compare.png}
%   \caption{例題2-2:matched/mismatched における $\mathrm{CR}_{00}$ の比較(B/G/R および白色).}
%   \label{fig:ex22_cr00_compare}
% \end{figure}
