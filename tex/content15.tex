\subsubsection{探索範囲と最適解(Case-1: LC/A/C, LC吸収軸基準)}
Case-1 の膜厚(位相差)最適化では,A-plate は LC の基準位相差 $\mathrm{Re}_{A,\mathrm{base}}$ を
\(\mathrm{Re}_{A,\mathrm{base}}=\mathrm{Re}_{\mathrm{LC}}/2\) として定義し,スケール係数 $A_{\mathrm{scale}}$ により
\(\mathrm{Re}_{A}=A_{\mathrm{scale}}\mathrm{Re}_{A,\mathrm{base}}\) と与える(本実装では \(\mathrm{Re}_{\mathrm{LC}}=310\,\mathrm{nm}\) より \(\mathrm{Re}_{A,\mathrm{base}}=155\,\mathrm{nm}\))。:contentReference[oaicite:0]{index=0}
探索はグリッドサーチで行い,A-plate は
\(A_{\min}=0.60,\ A_{\max}=1.40,\ \Delta A=0.05\),
C-plate は \(\mathrm{Re}_{C,\min}=-280\,\mathrm{nm},\ \mathrm{Re}_{C,\max}=+280\,\mathrm{nm},\ \Delta \mathrm{Re}_{C}=20\,\mathrm{nm}\)
の範囲で評価した。:contentReference[oaicite:1]{index=1}

この探索の結果,Case-1 の最適解は
\[
A_{\mathrm{scale}}=0.75,\qquad \mathrm{Re}_{C}=+80\,\mathrm{nm}
\]
であった(最適値を用いたスタック・ストークス追跡の設定にも同値が反映されている)。:contentReference[oaicite:2]{index=2}
したがって A-plate の位相差は
\(\mathrm{Re}_{A}=0.75\times 155=116.25\,\mathrm{nm}\) となる。:contentReference[oaicite:3]{index=3}

膜厚は \(\mathrm{Re}=\Delta n\,d\) より \(d[\mu\mathrm{m}]=\mathrm{Re}[\mathrm{nm}]/(1000\,\Delta n)\) で換算でき,
A-plate では(upper を用いる場合)\(\Delta n_{A}=\texttt{dn\_upperA}=0.00142\),
C-plate では \(\Delta n_{C}=\texttt{dn\_C}=0.12049\) を用いる。:contentReference[oaicite:4]{index=4}
さらに C-plate は \(\mathrm{Re}_{C}\) の符号を \(\Delta n_C\) の符号に持たせる実装であり,膜厚自体は
\(d_{C}=\left|\mathrm{Re}_{C}\right|/(1000\,|\Delta n_C|)\) で与えられる。:contentReference[oaicite:5]{index=5}

以上より(upper A を仮定すると),
\[
d_A \approx \frac{116.25}{1000\times 0.00142}=81.9\,\mu\mathrm{m},\qquad
d_C \approx \frac{80}{1000\times 0.12049}=0.664\,\mu\mathrm{m}
\]
となる。


\subsection{Case-1:LC/A/C(LC吸収軸基準)}
\subsubsection{スタック定義と最適化変数}
Case-1 では、LC を「吸収軸基準(absorption-basis)」で配置し、その後段に A-plate と C-plate を積層する:
\[
  \mathrm{POL}\ /\ \mathrm{LC}\ /\ \mathrm{A}\ /\ \mathrm{C}\ /\ \mathrm{Analyzer}.
\]
A-plate は upper/lower を選択可能だが、本ケースでは upper を固定し、A 軸の基準角は $90^\circ$ に固定して探索する。
探索変数は (i) $A_{\mathrm{scale}}$(A リタのスケール)、(ii) $\mathrm{Re}_C$(C リタ、符号付き)である。

\subsubsection{最良解の膜厚・リタデーション・光学軸設定}
out\_opt 内の最良解(\texttt{best\_stack\_case1.json} および \texttt{summary.csv})より、
Case-1 の最良パラメータは以下であった:
\begin{itemize}
  \item $A_{\mathrm{scale}}=0.75$(A-kind: upper, $A_{\mathrm{base}}=90^\circ$)
  \item $\mathrm{Re}_C=+80~\mathrm{nm}$
\end{itemize}
A-plate の各 1 枚当たりリタデーションは
$\mathrm{Re}_{A}=0.75\times 155=116.25~\mathrm{nm}$ であり、
$\Delta n_{A,\mathrm{up}}=0.00142$ より膜厚は
\[
  d_A=\frac{116.25}{1000\times 0.00142}\approx 81.9~\mu\mathrm{m}.
\]
C-plate は式 \eqref{eq:d_from_Re} により
\[
  d_C=\frac{80}{1000\times 0.12049}\approx 0.664~\mu\mathrm{m}
\]
となる。LC は固定値 $d_{\mathrm{LC}}=3.1~\mu\mathrm{m}$、$\Delta n_{\mathrm{LC}}=0.10$($\mathrm{Re}_{\mathrm{LC}}=310~\mathrm{nm}$)を用いる。

\begin{table}[t]
  \centering
  \caption{Case-1(LC/A/C, abs-basis)の最良解パラメータ。}
  \label{tab:best_case1}
  \begin{tabular}{lcccc}
    \toprule
    Layer & $\Delta n$ & $d~[\mu\mathrm{m}]$ & $\mathrm{Re}~[\mathrm{nm}]$ & Optical axis (in-plane) \\
    \midrule
    LC & $0.10$ & $3.1$ & $310$ & abs-basis($0^\circ$)+微小オフセット \\
    A (upper) & $0.00142$ & $81.9$ & $116.25$ & $A_{\mathrm{base}}=90^\circ$ +微小オフセット \\
    C & $0.12049$ & $0.664$ & $80$ & $z$ 軸(面外)\\
    \bottomrule
  \end{tabular}
\end{table}

\subsubsection{CR 等高線図と $\mathrm{CR}_{00}$}
Case-1 のベースライン(補償なし:LC のみ)に対し、最良解では
$(\theta,\phi)=(30^\circ,45^\circ)$ の CR が
\[
  107.86 \ \rightarrow\ 2530.68
\]
へと大幅に改善した。
一方で正面視の $\mathrm{CR}_{00}$ は
\[
  19752.8 \ \rightarrow\ 25212.6
\]
と増加したが、中心暗の改善量は斜視角ほど支配的ではない。
また、視野角格子全体の最小 CR(\texttt{global\_minCR})および 5\% 点(\texttt{global\_p5CR})も改善し、
暗状態の「底上げ」が確認できる。

% --- 図の後挿入(ファイル名のみ記す)---
% \begin{figure}[t]
%   \centering
%   \includegraphics[width=0.72\linewidth]{out_opt/iso_1_best.png}
%   \caption{Case-1 最良解の CR 等高線図(out\_opt/iso\_1\_best.png)。}
%   \label{fig:iso_case1_best}
% \end{figure}

\subsubsection{Stokes パラメータによる解釈と検算(case1\_results への言及)}
最良解において、$(30^\circ,45^\circ)$ の白色合成 Stokes は以下であった(代表):
\begin{table}[t]
  \centering
  \caption{Case-1 最良解の Stokes 追跡($(\theta,\phi)=(30^\circ,45^\circ)$)。}
  \label{tab:stokes_case1}
  \begin{tabular}{lrrrr}
    \toprule
    Stage & $s_1$ & $s_2$ & $s_3$ & $\psi~[\deg]$ \\
    \midrule
    POL\_in & $1.000$ & $0.000$ & $0.000$ & $0.00$ \\
    LC & $0.9999$ & $0.0139$ & $-0.0033$ & $0.40$ \\
    LC/A & $0.9308$ & $0.2410$ & $-0.2615$ & $7.26$ \\
    LC/A/C & $0.9654$ & $0.2460$ & $-0.0151$ & $7.15$ \\
    \bottomrule
  \end{tabular}
\end{table}
ここで $\psi=\tfrac{1}{2}\tan^{-1}(s_2/s_1)$ は線偏光の方位角(近似)である。
LC/A の段階で $|s_3|$ が増大し楕円偏光化するが、C-plate 追加後に $s_3$ が再び小さくなっており、
面外補償(C)により楕円率が抑制され、解析器吸収軸に対してより直交に近い線偏光へ整形されることが示唆される。

さらに \texttt{case1\_results\_20260108\_152708.txt} では、
transverse basis 上の解析器吸収軸方位角 $\alpha$ を幾何的に算出し、
式 \eqref{eq:Tleak_pred_from_stokes} により $T_{\mathrm{leak,pred}}$ を求めることで、
数値計算から得られる $T_{\mathrm{leak}}$ と整合することを確認している。
本ケースでも、最終段(LC/A/C)の Stokes を用いた推定により、
「解析器に対しほぼ直交となる偏光状態の実現」が CR 改善の直接要因であることを説明できる。

\subsubsection{進捗(progress)}
探索の進捗は \texttt{progress\_case1.csv} に記録され、
更新番号(\texttt{update\_idx})ごとに最良 CR(\texttt{best\_CR})が単調に更新される。
また、各更新点で \texttt{s1\_el\#0\_LC} 等の Stokes を併記することで、
「どの層追加が楕円率や方位角に寄与し、最終的に漏れを抑えたか」を追跡可能としている。

% \begin{figure}[t]
%   \centering
%   \includegraphics[width=0.72\linewidth]{out_opt/progress_case1_CR.png}
%   \caption{Case-1 の探索進捗(out\_opt/progress\_case1\_CR.png)。}
%   \label{fig:progress_case1}
% \end{figure}

\subsection{Case-2:C/A/LC(LC透過軸基準)}
\subsubsection{スタック定義と最適化変数}
Case-2 では、C-plate と A-plate を LC の前段に配置し、LC は「透過軸基準(transmission-basis)」で配置する:
\[
  \mathrm{POL}\ /\ \mathrm{C}\ /\ \mathrm{A}\ /\ \mathrm{LC}\ /\ \mathrm{Analyzer}.
\]
本実装では A-kind(upper/lower)と A 軸基準角($0^\circ$ または $90^\circ$)を候補として持ち、
$A_{\mathrm{scale}}$ と $\mathrm{Re}_C$ を含めて最良解を探索する。

\subsubsection{最良解の膜厚・リタデーション・光学軸設定}
out\_opt 内の Case-2 対応結果(本実行では \texttt{case3} として出力される)より、最良パラメータは
\begin{itemize}
  \item $A_{\mathrm{scale}}=0.80$(A-kind: upper, $A_{\mathrm{base}}=0^\circ$)
  \item $\mathrm{Re}_C=+90~\mathrm{nm}$
\end{itemize}
であった。A-plate の各 1 枚当たりリタは $\mathrm{Re}_A=124.0~\mathrm{nm}$ であり、
\[
  d_A=\frac{124.0}{1000\times 0.00142}\approx 87.3~\mu\mathrm{m},
  \qquad
  d_C=\frac{90}{1000\times 0.12049}\approx 0.747~\mu\mathrm{m}.
\]
LC は透過軸基準(in-plane $90^\circ$)で固定パラメータを用いる。

\begin{table}[t]
  \centering
  \caption{Case-2(C/A/LC, tran-basis)の最良解パラメータ。}
  \label{tab:best_case2}
  \begin{tabular}{lcccc}
    \toprule
    Layer & $\Delta n$ & $d~[\mu\mathrm{m}]$ & $\mathrm{Re}~[\mathrm{nm}]$ & Optical axis (in-plane) \\
    \midrule
    C & $0.12049$ & $0.747$ & $90$ & $z$ 軸(面外)\\
    A (upper) & $0.00142$ & $87.3$ & $124.0$ & $A_{\mathrm{base}}=0^\circ$ +微小オフセット \\
    LC & $0.10$ & $3.1$ & $310$ & tran-basis($90^\circ$)+微小オフセット \\
    \bottomrule
  \end{tabular}
\end{table}

\subsubsection{CR 等高線図と $\mathrm{CR}_{00}$}
Case-2(出力上は \texttt{case3})のベースライン(補償なし:LC のみ)に対し、
$(30^\circ,45^\circ)$ の CR は
\[
  89.01 \ \rightarrow\ 1493.64
\]
へ改善した。また $\mathrm{CR}_{00}$ は
\[
  136600.8 \ \rightarrow\ 20192.9
\]
となり、斜視角改善と引き換えに正面暗が低下するトレードオフが観測される。
一方で、視野角格子全体の \texttt{global\_minCR}, \texttt{global\_p5CR} は改善しており、
「斜視角のボトム改善」を優先する設計方針としては有効である。

% --- 図の後挿入(ファイル名のみ記す)---
% \begin{figure}[t]
%   \centering
%   \includegraphics[width=0.72\linewidth]{out_opt/iso_3_best.png}
%   \caption{Case-2(出力上は case3)最良解の CR 等高線図(out\_opt/iso\_3\_best.png)。}
%   \label{fig:iso_case2_best}
% \end{figure}

\subsubsection{Stokes パラメータによる解釈}
Case-2 最良解の Stokes 追跡($(30^\circ,45^\circ)$)は以下の通りである:
\begin{table}[t]
  \centering
  \caption{Case-2 最良解の Stokes 追跡($(\theta,\phi)=(30^\circ,45^\circ)$)。}
  \label{tab:stokes_case2}
  \begin{tabular}{lrrrr}
    \toprule
    Stage & $s_1$ & $s_2$ & $s_3$ & $\psi~[\deg]$ \\
    \midrule
    POL\_in & $1.000$ & $0.000$ & $0.000$ & $0.00$ \\
    C & $0.9602$ & $-0.0057$ & $0.2769$ & $-0.17$ \\
    C/A & $0.9578$ & $0.2707$ & $0.0231$ & $7.89$ \\
    C/A/LC & $0.9533$ & $0.2857$ & $-0.0655$ & $8.34$ \\
    \bottomrule
  \end{tabular}
\end{table}
C 単体で楕円率($|s_3|$)が増加する一方、A により $s_3$ が抑制され線偏光へ近づく。
最終的に LC を透過軸基準で置くことで、解析器吸収軸に対する偏光方位角が最適化され、
所望視角点での漏れが低減する。

\subsubsection{進捗(progress)}
探索の進捗は \texttt{progress\_case3.csv}(Case-2 相当)に保存される。
ここには、更新ごとの \texttt{best\_CR} に加え、\texttt{s1\_el\#0\_C} 等として
C/A/LC 各段階の Stokes が記録され、どの層が楕円率($s_3$)や方位角($s_1,s_2$)に効いて
最終 CR 改善へ至ったかを定量的に追跡できる。

% \begin{figure}[t]
%   \centering
%   \includegraphics[width=0.72\linewidth]{out_opt/progress_case3_CR.png}
%   \caption{Case-2(case3)の探索進捗(out\_opt/progress\_case3\_CR.png)。}
%   \label{fig:progress_case2}
% \end{figure}

\subsection{小括}
(case1) LC/A/C(LC 吸収軸基準)では、C により楕円率($s_3$)が補償され、
解析器吸収軸に対してより直交に近い偏光状態を作ることで、所望視角点の CR を大きく改善できた。
(case2) C/A/LC(LC 透過軸基準)でも同様に所望視角点の CR は改善するが、
正面暗($\mathrm{CR}_{00}$)が低下し得るため、視角優先の設計か正面暗優先かに応じた
目的関数(例えば複数視角点の同時最適化、あるいは $\mathrm{CR}_{00}$ へのペナルティ付与)の再設計が必要である。
