\section{波長依存性評価}
本節では,最適化で得た補償条件(${\rm ReA}=116.25~{\rm nm}$,${\rm ReC}=85~{\rm nm}$)を固定したまま,
単色波長 $\lambda$ を掃引し,(i) 正面暗状態 ${\rm CR}00(\lambda)$ と,
(ii) 斜め視点 $\theta=30^\circ$ における 4方位平均
\[
\overline{{\rm CR}}_{30}(\lambda)
\equiv
\frac{1}{4}\Bigl(
{\rm CR}(30,+45;\lambda)+{\rm CR}(30,-45;\lambda)+{\rm CR}(30,+135;\lambda)+{\rm CR}(30,-135;\lambda)
\Bigr)
\]
の波長依存性を評価する.ここで ${\rm CR}00(\lambda)$ は単色(mono)評価,
$\overline{{\rm CR}}_{30}(\lambda)$ も同様に単色(mono)評価である.
また ISO 分布として,代表波長 B/G/R(450/546/610~nm)の単色 ISO と,
B/G/R のリーク透過率平均に基づく白色(W)ISO を併記し,波長による分布変化を視覚的に示す.

\subsection{評価条件}
波長掃引は
\[
\lambda\in\{450,470,\ldots,650\}~{\rm nm}\qquad(\Delta\lambda=20~{\rm nm})
\]
で行い,分散モデルは \texttt{flat}(材料の $\Delta n(\lambda)$ の分散を入れず,
主として位相遅れが $\Gamma(\lambda)\propto 1/\lambda$ により波長依存するモデル)を用いた.
このとき短波長側ほど位相回転量が大きくなり,補償条件が波長とともに過補償/不足補償へ移行し得る.

\subsection{結果}
表\ref{tab:disp_summary}に,波長ごとの ${\rm CR}00(\lambda)$ と
4方位平均 $\overline{{\rm CR}}_{30}(\lambda)$,
ならびに内訳として ${\rm CR}(30,\phi;\lambda)$($\phi=\pm45^\circ,\pm135^\circ$)を示す.

\begin{table}[htb]
\centering
\caption{波長掃引における単色評価(mono)のまとめ(\texttt{dispersion\_summary\_CRavg4.csv} 相当).
${\rm CR}00(\lambda)$ と $\overline{{\rm CR}}_{30}(\lambda)$,
および $\theta=30^\circ$・$\phi=\pm45^\circ,\pm135^\circ$ の各 ${\rm CR}$ を示す.}
\label{tab:disp_summary}
\setlength{\tabcolsep}{4pt}
\small
\begin{tabular}{r|r|r|rrrr}
\hline
$\lambda$ [nm] &
${\rm CR}00$ &
$\overline{{\rm CR}}_{30}$ &
${\rm CR}(30,+45)$ &
${\rm CR}(30,-45)$ &
${\rm CR}(30,+135)$ &
${\rm CR}(30,-135)$ \\
\hline
450 & 10800.9 &  4912.95 &  2702.37 &  7123.54 &  7123.54 &  2702.37 \\
470 & 11048.7 &  8367.11 & 11739.44 &  4994.78 &  4994.78 & 11739.44 \\
490 & 11335.0 &  8814.77 & 15600.25 &  2029.29 &  2029.29 & 15600.25 \\
510 & 11652.8 &  2656.84 &  4251.86 &  1061.82 &  1061.82 &  4251.86 \\
530 & 11996.5 &  1252.88 &  1835.71 &   670.04 &   670.04 &  1835.71 \\
550 & 12361.8 &   762.46 &  1049.99 &   474.93 &   474.93 &  1049.99 \\
570 & 12745.2 &   531.83 &   700.44 &   363.21 &   363.21 &   700.44 \\
590 & 13143.9 &   403.11 &   513.43 &   292.79 &   292.79 &   513.43 \\
610 & 13555.3 &   323.00 &   400.77 &   245.20 &   245.20 &   400.77 \\
630 & 13977.4 &   269.21 &   327.10 &   211.33 &   211.33 &   327.10 \\
650 & 14408.4 &   231.08 &   275.94 &   186.22 &   186.22 &   275.94 \\
\hline
\end{tabular}
\end{table}

図\ref{fig:lambda_dep}は ${\rm CR}00(\lambda)$ と
$\overline{{\rm CR}}_{30}(\lambda)$ の波長依存性を示す

\begin{figure}[htb]
 \centering
 \includegraphics[width=0.85\linewidth]{plot_lambda_vs_CR00_and_CRavg4_mono.png}
 \caption{単色評価(mono)における ${\rm CR}00(\lambda)$ と
 4方位平均 $\overline{{\rm CR}}_{30}(\lambda)$ の波長依存性.}
 \label{fig:lambda_dep}
\end{figure}

図\ref{fig:lambda_dep}(および表\ref{tab:disp_summary})から,
${\rm CR}00(\lambda)$ と $\overline{{\rm CR}}_{30}(\lambda)$ はいずれも顕著な波長依存性を持つが,
依存の傾向は互いに大きく異なることが分かる.
具体的には,正面暗状態は
\[
{\rm CR}00(450~{\rm nm})\approx 1.08\times 10^4 \ \rightarrow\
{\rm CR}00(650~{\rm nm})\approx 1.44\times 10^4
\]
と長波長側で緩やかに増加するのに対し,
斜め視点の平均性能は
\[
\overline{{\rm CR}}_{30}(490~{\rm nm})\approx 8.8\times 10^3
\]
付近で最大となった後,
長波長側で急激に低下し,
\[
\overline{{\rm CR}}_{30}(650~{\rm nm})\approx 2.3\times 10^2
\]
まで落ちる.
すなわち,正面暗状態を改善する方向(長波長側)と,
斜め視点の平均性能を改善する方向(短波長側)とが一致しない.

この差は,補償スタックが与える偏光状態の回転量(位相遅れ)が $\Gamma(\lambda)\propto 1/\lambda$ として
波長により変化することに起因し,(i) 正面($\theta=0$)では残留位相が小さくなるほど暗状態リークが減少しやすい一方,
(ii) 斜め視点($\theta=30^\circ$)では波長により「直交条件が回復する方位と程度」が強く変化するためである.
このため,広帯域で両者を同時に改善するには,
実材料の分散を考慮した補償設計か,A-plateなどの光学フィルムの追加が必要となる.

\subsection{RGBW毎の CR視野角分布(等高線ISO)}
代表波長 B/G/R(450/546/610~nm)の単色と白色(W)の
CR視野角分布(ISO) を図\ref{fig:iso_rgbw}に示す.
ここで W は,B/G/R のリーク透過率 $T_{\rm leak}$ を平均して
白色リークを作る定義であり,最適化で用いた W 評価と整合する.
一方,B/G/R は単色(mono)ISO であり,
波長ごとの分布変化(ローブの回転・入れ替わり)を直観的に把握するために用いる.

\begin{figure}[H]
  \centering
  % (a) の図
  \begin{subfigure}{0.48\linewidth}
    \caption{}
    \centering
   \includegraphics[width=\linewidth]{iso_B_mono_450nm.png}
  \end{subfigure}
  \hfill % 左右の図の間に適切な空間を挿入
  % (b) の図
  \begin{subfigure}{0.48\linewidth}
    \caption{}
    \centering
    \includegraphics[width=\linewidth]{iso_G_mono_546nm.png}
  \end{subfigure}
  \hfill % 左右の図の間に適切な空間を挿入
  % (c) の図
  \begin{subfigure}{0.48\linewidth}
    \caption{}
    \centering
    \includegraphics[width=\linewidth]{iso_R_mono_610nm.png}
  \end{subfigure}
  % (d) の図
  \begin{subfigure}{0.48\linewidth}
    \caption{}
    \centering
    \includegraphics[width=\linewidth]{iso_W.png}
  \end{subfigure}
 \caption{代表波長 B/G/R の単色 ISO と,白色(W)ISO の比較.
 単色では波長により ISO 分布(高 ${\rm CR}$ ローブの位置・強度)が大きく変化し得るのに対し,
 W は B/G/R のリーク平均によりスペクトル平均化された分布を与える.}
 \label{fig:iso_rgbw}
\end{figure}

% \begin{figure}[t]
% \centering
% \begin{minipage}{0.48\linewidth}
%   \centering
%   \includegraphics[width=\linewidth]{out_disp/iso_B_mono_450nm.png}
%   \caption*{(a) B: 450~nm (mono)}
% \end{minipage}\hfill
% \begin{minipage}{0.48\linewidth}
%   \centering
%   \includegraphics[width=\linewidth]{out_disp/iso_G_mono_546nm.png}
%   \caption*{(b) G: 546~nm (mono)}
% \end{minipage}
%
% \vspace{0.8em}
% \begin{minipage}{0.48\linewidth}
%   \centering
%   \includegraphics[width=\linewidth]{out_disp/iso_R_mono_610nm.png}
%   \caption*{(c) R: 610~nm (mono)}
% \end{minipage}\hfill
% \begin{minipage}{0.48\linewidth}
%   \centering
%   \includegraphics[width=\linewidth]{out_disp/iso_W.png}
%   \caption*{(d) W: B/G/R averaged (Tleak-avg)}
% \end{minipage}
%
% \caption{代表波長 B/G/R の単色 ISO と,白色(W)ISO の比較.
% 単色では波長により ISO 分布(高 ${\rm CR}$ ローブの位置・強度)が大きく変化し得るのに対し,
% W は B/G/R のリーク平均によりスペクトル平均化された分布を与える.}
% \label{fig:iso_rgbw}
% \end{figure}

図\ref{fig:iso_rgbw}では,B(450~nm)と G(546~nm),R(610~nm)で
高 ${\rm CR}$ のローブ構造が異なり,特に短波長側で分布の変化が大きいことが分かる.
これは表\ref{tab:disp_summary}に見られるように,
$\theta=30^\circ$ の特定方位で ${\rm CR}$ が急激に増減し得ることと整合する.
一方で W(白色)は B/G/R を平均化するため,
単色の鋭いピークや方位入れ替わりは緩和され,
広帯域で観測される見かけの視野角特性に近い分布となる.

\subsection{実材料分散と分散補償スタックの必要性}
本節の結果は \texttt{flat} 分散($\Delta n(\lambda)$ の分散を無視)に基づくため,
実材料(TAC,A-plate,C-plate,LC)の波長分散を考慮すると,
\[
\Gamma(\lambda)\propto \frac{\Delta n(\lambda)\,d}{\lambda}
\]
により波長依存性はさらに複雑化し得る.
特に LC の複屈折分散やポリマー補償膜の分散は,
単色最適化条件を白色へ拡張する際の主要因となる.
したがって,広帯域で安定な視野角補償(${\rm CR}$ の波長依存と方位非対称の抑制)を実現するには,
実材料の分散を組み込んだモデルに基づき,
\begin{itemize}
\item A/C の材料分散を反映した ${\rm ReA}(\lambda)$,${\rm ReC}(\lambda)$ の整合,
\item 追加補償板(多層 A/C,あるいは分散補償用の複合膜)による波長依存性の相殺,
\item 目的関数として白色(W)だけでなく,単色B/G/Rの最悪値や方位非対称も含めた同時最適化
\end{itemize}
など,波長依存性を消す(または抑える)ような補償フィルム構成が必要である.
