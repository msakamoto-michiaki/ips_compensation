\subsection{例題2:A-plate/C-plate 膜厚の同時スイープによる暗状態視野角最適化(\texttt{ex2.py})}
本例題では,A-plate と C-plate の膜厚(本実装では A-plate はスケール因子 $A_{\mathrm{scale}}$,C-plate は符号付き膜厚 $C_{\mu\mathrm{m}}$)を二次元で走査し,暗状態コントラストの \textbf{正面性能}と\textbf{視野角性能}を同時に評価する.評価は白色(B/G/R 平均)かつ分散整合(matched)条件で行い,$\theta_{\max}=60^\circ$,角度刻み $\Delta\theta=\Delta\phi=5^\circ$ の格子上で $\mathrm{CR}(\theta,\phi)$ を計算した(\texttt{examples/ex2\_Ascale\_C\_grid\_check\_signedC/} 以下に出力).本節では,(i) 正面コントラスト $\mathrm{CR}_{00}$,(ii) しきい値超過率 $\mathrm{frac}\{\mathrm{CR}>\mathrm{thr}\}$,(iii) ISO 等高線(ISO\_PLOT)の三点から最適条件を説明する.

\subsubsection{評価指標}
\paragraph{(1) 正面コントラスト $\mathrm{CR}_{00}$}
正面 $(\theta,\phi)=(0,0)$ における漏れ透過率 $T_{\mathrm{leak}}$ から
\[
\mathrm{CR}_{00}=\frac{1}{T_{\mathrm{leak}}(0,0)}
\]
として定義する(\texttt{CR\_from\_Tleak}).本スタックでは C-plate の光学軸が $z$ 方向($[0,0,1]$)であり,正面入射では波数ベクトル $\mathbf{k}$ と平行となるため C-plate の位相遅れ寄与が消失する.したがって,\textbf{$\mathrm{CR}_{00}$ は $C_{\mu\mathrm{m}}$ に依存せず,主として $A_{\mathrm{scale}}$ により決まる}.実際,今回の走査では
\[
A_{\mathrm{scale}}=0.5,\,1.0,\,1.5,\,2.0 \quad \Rightarrow \quad
\mathrm{CR}_{00}\approx 2.39\times 10^3,\ 4.49\times 10^3,\ 1.26\times 10^4,\ 1.21\times 10^4
\]
となり,正面最適は $A_{\mathrm{scale}}\simeq 1.5$ 付近に現れる.

\paragraph{(2) しきい値超過率 $\mathrm{frac}\{\mathrm{CR}>\mathrm{thr}\}$}
視野角格子上で $\mathrm{CR}(\theta,\phi)>\mathrm{thr}$ を満たす領域の割合を,立体角重み $\sin\theta$ を用いて
\[
\mathrm{frac}\{\mathrm{CR}>\mathrm{thr}\}
=\frac{\sum_{\theta,\phi}\sin\theta\,\Delta\theta\,\Delta\phi\ \mathbb{I}\!\left(\mathrm{CR}(\theta,\phi)>\mathrm{thr}\right)}
{\sum_{\theta,\phi}\sin\theta\,\Delta\theta\,\Delta\phi}
\]
で近似評価する(実装は \texttt{\_solid\_angle\_fraction\_over\_threshold}).本例題では $\mathrm{thr}=100$ とし,\textbf{視野角で暗状態が維持される範囲の広さ}を単一スカラーで比較する.

\paragraph{(3) ISO\_PLOT(等高線形状)}
$\mathrm{CR}(\theta,\phi)$ の等高線を極座標上に描き,$\mathrm{CR}=100$ 等の境界線の張り出し(到達 $\theta$)と形状の歪み(方位依存)を可視化する.本例題では塗りつぶしを行わず,等高線のみを描画した(\texttt{ISO\_contour\_matched\_C\_d*.png}).

\subsubsection{A-plate 膜厚($A_{\mathrm{scale}}$)の最適化:$\mathrm{CR}_{00}$ に基づく候補選定}
A-plate は基準 retardation(各 A-plate の基準値 $R_{0,\mathrm{A}}^{\mathrm{base}}=155\ \mathrm{nm}$)を $A_{\mathrm{scale}}$ 倍して設定する.膜厚は $\Delta n$ により
\[
d_{\mathrm{A,bot}}=\frac{R_{0,\mathrm{A}}^{\mathrm{base}}A_{\mathrm{scale}}}{\Delta n_{\mathrm{A,bot}}},\qquad
d_{\mathrm{A,top}}=\frac{R_{0,\mathrm{A}}^{\mathrm{base}}A_{\mathrm{scale}}}{\Delta n_{\mathrm{A,top}}}
\]
で与えられる(本実装では $\Delta n_{\mathrm{A,bot}}=0.00145,\ \Delta n_{\mathrm{A,top}}=0.00142$).例えば
\[
A_{\mathrm{scale}}=2.0 \Rightarrow
(d_{\mathrm{A,bot}},d_{\mathrm{A,top}})\approx(214,\ 218)\ \mu\mathrm{m},
\qquad
A_{\mathrm{scale}}=1.0 \Rightarrow (107,\ 109)\ \mu\mathrm{m}
\]
となる.
前述の通り $\mathrm{CR}_{00}$ は $A_{\mathrm{scale}}$ に強く依存し,$A_{\mathrm{scale}}\approx 1.5$ で最大($\sim 1.26\times 10^4$)となった.一方で,$A_{\mathrm{scale}}=2.0$ でも $\mathrm{CR}_{00}\sim 1.2\times 10^4$ を維持しており,\textbf{正面性能の観点では $A_{\mathrm{scale}}=1.5\sim 2.0$ が実用的な候補域}といえる.

\subsubsection{C-plate 膜厚($C_{\mu\mathrm{m}}$)の最適化:$\mathrm{frac}\{\mathrm{CR}>100\}$ と ISO\_PLOT に基づく決定}
C-plate は符号付き膜厚 $C_{\mu\mathrm{m}}$ で与えられ,\textbf{$C_{\mu\mathrm{m}}>0$ の場合は上下に同厚の C-plate(L-C と U-C)を配置し,$C_{\mu\mathrm{m}}<0$ の場合は U-C のみを配置して複屈折符号を反転($-\Delta n_{\mathrm{C}}$)}する(\texttt{signedC} 実装).したがって,$C_{\mu\mathrm{m}}=+0.5$ は上下それぞれ $0.5\ \mu\mathrm{m}$(合計 $1.0\ \mu\mathrm{m}$ 相当),$C_{\mu\mathrm{m}}=+1.0$ は合計 $2.0\ \mu\mathrm{m}$ 相当の C-plate を意味する.

二次元走査($A_{\mathrm{scale}}\in\{0.5,1.0,1.5,2.0\}$,$C_{\mu\mathrm{m}}\in\{-1.5,-1.0,-0.5,0,0.5,1.0,1.5\}$)の結果,$\mathrm{frac}\{\mathrm{CR}>100\}$ は $C_{\mu\mathrm{m}}$ に対して明瞭な最適点を示し,その最適 $C_{\mu\mathrm{m}}$ は $A_{\mathrm{scale}}$ によりシフトした.具体的には,
\[
A_{\mathrm{scale}}=1.0 \Rightarrow C_{\mu\mathrm{m}}=+1.0\ \text{で最大} \ (\mathrm{frac}\approx 0.530),
\qquad
A_{\mathrm{scale}}=2.0 \Rightarrow C_{\mu\mathrm{m}}=+0.5\ \text{で最大} \ (\mathrm{frac}\approx 0.483)
\]
となり,\textbf{A-plate を厚くすると最適 C-plate は薄い側へ移る}傾向が確認された.ISO\_PLOT でも,最適近傍では $\mathrm{CR}=100$ 等高線がより大きな $\theta$ まで張り出し,方位方向に極端な切れ込みが生じにくい(視野角全体での暗状態維持)ことが視覚的に確認できる.逆に,負の $C_{\mu\mathrm{m}}$($-\Delta n_{\mathrm{C}}$)や過大な $|C_{\mu\mathrm{m}}|$ では,高 $\theta$ 側で等高線が内側へ退き,$\mathrm{frac}\{\mathrm{CR}>100\}$ が低下した.

以上を踏まえ,本例題では \textbf{正面性能($\mathrm{CR}_{00}$)と視野角性能($\mathrm{frac}$,ISO\_PLOT)の両立}という観点から,
\[
\boxed{
A_{\mathrm{scale}}=2.0,\quad C_{\mu\mathrm{m}}=+0.5
}
\]
を推奨解として採用する.この条件は $\mathrm{CR}_{00}\approx 1.21\times 10^4$ を維持しつつ,$\mathrm{frac}\{\mathrm{CR}>100\}\approx 0.483$ と大きな視野角領域を確保する(図は別途挿入).一方,\textbf{視野角領域の最大化のみ}を主目的とする場合は $(A_{\mathrm{scale}},C_{\mu\mathrm{m}})=(1.0,+1.0)$ が $\mathrm{frac}\approx 0.530$ と最大であり,設計意図に応じて選択すべきである.

% % ---- 図(別途貼り付け) ----
% \begin{figure}[t]
%   \centering
%   % \includegraphics[width=0.48\linewidth]{ISO_contour_matched_C_d1.0um_A1.00_thr100.png}
%   % \includegraphics[width=0.48\linewidth]{ISO_contour_matched_C_d0.5um_A2.00_thr100.png}
%   \caption{例題2:$A_{\mathrm{scale}}$ と $C_{\mu\mathrm{m}}$ の走査による ISO 等高線($\mathrm{thr}=100$)の比較例.
%   左:$\mathrm{frac}\{\mathrm{CR}>100\}$ 最大($A_{\mathrm{scale}}=1.0,\,C_{\mu\mathrm{m}}=+1.0$).
%   右:$\mathrm{CR}_{00}$ を高く保ちつつ視野角を広げた推奨条件($A_{\mathrm{scale}}=2.0,\,C_{\mu\mathrm{m}}=+0.5$).}
%   \label{fig:ex2_iso_compare}
% \end{figure}

\subsubsection{小括}
\texttt{ex2.py} による二次元走査は,(i) 正面指標 $\mathrm{CR}_{00}$ により $A_{\mathrm{scale}}$ の候補域を定め,(ii) $\mathrm{frac}\{\mathrm{CR}>\mathrm{thr}\}$ と ISO\_PLOT により $C_{\mu\mathrm{m}}$ を最適化する,という実務的な設計分解を与える.特に C-plate は正面には影響せず視野角で支配的に効くため,\textbf{「正面は A-plate,視野角は C-plate」}という役割分担が本結果からも確認できる.
