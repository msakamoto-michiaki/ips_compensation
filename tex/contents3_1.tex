\subsubsection{A-plate の基準値とスケール因子}
設計の第一段階では,LC 層を暗状態漏れの主因となる retarder とみなし,正面付近(小 $\theta$)で支配的な面内成分を代表量 $R_{0,\mathrm{LC}}$ により記述する.このとき基本方針は,\textbf{上下 2 枚の A-plate を合成 retarder として用い,LC が与える面内 retardation に対抗する retardation を導入することで,光学軸ずれに対する消光条件の感度を低減する}ことである.概念的には,暗漏れに寄与する偏光状態変化に対して有効に投影された成分の意味で,
\[
R_{0,\mathrm{A,top}}^{(\mathrm{eff})}+R_{0,\mathrm{A,bot}}^{(\mathrm{eff})}\approx R_{0,\mathrm{LC}}
\]
を基準関係として置く.ここで $(\mathrm{eff})$ は,各層の光軸方位および配置を考慮した「暗漏れに対する有効寄与」を表す概念量であり,位相遅れの大きさと主軸方位の組合せにより定まる.この基準関係は,設計点(設計波長・正面付近)での暗漏れ抑制に加え,POL/A-plate/LC の相対軸が理想値からずれた場合でも,LC 単独による偏光変換が暗漏れへ変換される度合いを抑制する設計思想を表す.上下 A-plate で $A_{\mathrm{scale}}$ を共通とする場合は matched 条件の仮定に相当し,別個に与える場合はフィルム差・製造差等を含む一般化に相当する.

A-plate の設計は,基準波長(典型的には G)における基準面内リタデーション $R_{0,\mathrm{A}}^{\mathrm{ref}}$ を設定し,これに対するスケール因子 $A_{\mathrm{scale}}$ を導入して
\[
R_{0,\mathrm{A}}(\lambda)=A_{\mathrm{scale}}\;R_{0,\mathrm{A}}^{\mathrm{ref}}(\lambda)
\]
の形で最適化を行う($R_{0,\mathrm{A}}^{\mathrm{ref}}(\lambda)$ は基準分散を含む既知プロファイルとする).上下 A-plate で $A_{\mathrm{scale}}$ を共通とする場合は matched 条件の仮定に相当し,別個に与える場合はフィルム差・製造差等を含む一般化に相当する.

\subsubsection{斜入射に対する拡張:C-plate による厚み方向成分の補償(高次設計)}
一次設計が成立しても,斜入射では実効複屈折が変化し,光学軸の見かけ(投影)も変化するため,暗状態漏れが再増大し得る.また,斜入射では厚み方向成分が顕在化し,面内補償のみでは抑制が困難な残差が生じる.このため C-plate を導入し,厚み方向リタデーション $R_{\mathrm{th}}$ により斜入射で増大する残差位相を相殺する.概念的には,角度依存計算により得られる残差厚み成分 $R_{\mathrm{th}}^{(\mathrm{res})}(\theta,\phi)$ に対して,
\[
R_{\mathrm{th},\mathrm{C}}^{(\mathrm{eff})}\approx -\,R_{\mathrm{th}}^{(\mathrm{res})}(\theta,\phi)
\]
となるように $R_{\mathrm{th},\mathrm{C}}$ を設定することで,広視野角領域における暗状態漏れの増大を抑制する.この効果は,所定閾値を満たす領域割合 $\mathrm{frac}\{\mathrm{CR}>\mathrm{thr}\}$ の改善として評価される.

\subsubsection{設計手順(概念的ワークフロー)}
retardation と光学軸ずれの観点から整理した設計手順は以下の通りである.
\begin{enumerate}
  \item \textbf{基準設定:} LC の代表面内リタデーション $R_{0,\mathrm{LC}}$(必要に応じて $R_{\mathrm{th},\mathrm{LC}}$)を設計波長で定義し,配向角誤差(相対光学軸ずれ)および視野角 $(\theta,\phi)$ を外乱として位置づける.
  \item \textbf{一次設計(面内整合):} 上下 A-plate の $R_0$ と配向角を選び,暗漏れに対する有効寄与の意味で $R_{0,\mathrm{A,top}}^{(\mathrm{eff})}+R_{0,\mathrm{A,bot}}^{(\mathrm{eff})}\approx R_{0,\mathrm{LC}}$ を満たす条件を基準として設定し,正面白色 $\mathrm{CR}_0$ を確保する.
  \item \textbf{角度依存残差の評価:} $R_i^{\mathrm{eff}}(\theta,\phi,\lambda)$ の角度・波長依存を評価し,面内成分と厚み方向成分に分解して残差を抽出する(詳細は数値計算で扱う).
  \item \textbf{高次設計(厚み方向補償):} C-plate の $R_{\mathrm{th}}$(必要に応じて $R_0$)を導入し,斜入射で顕在化する厚み方向残差を抑制するよう最適化し,$\mathrm{frac}\{\mathrm{CR}>\mathrm{thr}\}$ を改善する.
  \item \textbf{白色評価(分散整合):} LC と補償板の $\Delta n(\lambda)$ の相対スケール差(dispersion mismatch)が白色 $\mathrm{CR}_0$ の低下要因となるため,matched 条件を優先する等の制約を付与し,波長間で補償条件が過度に乖離しないよう設計する.
\end{enumerate}

\subsubsection{要約}
A-plate/C-plate/LC の補償設計は,(i) LC の面内リタデーションを基準量として,上下 A-plate の合成 retarder 効果により対抗的な面内 retardation を導入し,正面性能と光学軸ずれに対する感度を低減すること,(ii) 斜入射で顕在化する厚み方向残差に対して C-plate の $R_{\mathrm{th}}$ により補償を付加し,広視野角での漏れ増大を抑制すること,(iii) 白色評価では分散整合(matched / mismatched)が $\mathrm{CR}_0$ を強く左右するため,必要に応じて matched 条件を設計制約として組み込むこと,に要約される.
