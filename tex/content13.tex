\subsection{光学補償の設計思想:
$\sim$ 偏光板のクロスニコル配置の斜め視野からの崩れを Stokes/Poincar\'e 表現で捉え,光学補償フィルムで補償する $\sim$
}

\subsubsection{斜め視野における暗状態漏れの本質(クロスニコルの幾何学的崩れ)}
IPS の暗状態は,正面($\theta=0$)では入射偏光板(Polarizer)と出射偏光板(Analyzer)がクロスニコル(互いに $90^\circ$)となるように設定し,
入射偏光板透過後の光の直線偏光が Analyzer により完全に吸収されることで実現される.
しかし斜め入射では,光線ベクトル $\hat{\bm k}(\theta,\phi)$ に垂直な平面(横波面)上での
「偏光板の有効光軸(透過軸/吸収軸)」が正面の設定から偏位し,
クロス条件が見かけ上 $90^\circ$ から広がる(あるいは狭まる).
このとき,正面では一致していた「Analyzer の吸収軸」と「Polarizer 透過後偏光状態」が一致しなくなり,
暗状態漏れが生じる.とくに対角方向($\phi\simeq 45^\circ$)で漏れが最大化されやすい.

\subsubsection{Stokes ベクトルと「直線偏光化+方位合わせ」という設計目標}
横波面上で直交基底 $(\bm u,\bm v)$ を定義し,
各層通過後の偏光状態を規格化 Stokes ベクトル
$\bm s=(s_1,s_2,s_3)$($|\bm s|=1$)で表す.
$s_3$ は楕円偏光の楕円率(円偏光成分)を表し,$s_3\simeq 0$ は「ほぼ直線偏光(Poincar\'e 球の赤道)」を意味する.
また $(s_1,s_2)$ から,横波面内での直線偏光の方位角 $\psi$ は
\begin{equation}
  \psi = \frac{1}{2}\,\mathrm{atan2}(s_2,s_1)
\end{equation}
で与えられる($\psi$ は $(\bm u,\bm v)$ に対する角度).

一方,斜め入射では Analyzer の「透過軸」も横波面内で方位角 $\alpha(\theta,\phi)$ をもつ(正面の $90^\circ$ から偏位する).
直線偏光($s_3=0$)が方位 $\psi$ をもつとき,理想的なクロスニコルは
\begin{equation}
  \Delta(\theta,\phi)\equiv (\alpha-\psi)\bmod 180^\circ \simeq 90^\circ
\end{equation}
で表され,$\Delta$ が $90^\circ$ からずれるほど漏れが増える.
従って暗状態改善の設計目標は次の二点に要約できる:
\begin{enumerate}
\item \textbf{直線偏光化:} $\;s_3(\theta,\phi)\to 0$(楕円率を消す)
\item \textbf{方位合わせ:} $\;\Delta(\theta,\phi)\to 90^\circ$(出射偏光の方位を Analyzer の吸収方向へ整列)
\end{enumerate}
ここで重要なのは,正面の CR$_{00}$ を落とさないことである.
すなわち,補償スタックは $\theta=0$ において POL--Analyzer のクロス条件を壊さず,
同時に $\theta>0$ で生じる $\alpha(\theta,\phi)$ の偏位を「偏光状態の回転量」で打ち消すように働く必要がある.

\subsubsection{LC/A/C(LC吸収軸基準)の設計思想}
本研究の LC/A/C は,液晶LCの長軸方向を入射偏光板 POL の吸収軸にとり,
\begin{equation}
  \mathrm{POL}\; /\; \mathrm{LC}\; /\; +\mathrm{A}\; /\; \pm \mathrm{C}\; /\; \mathrm{Analyzer}
\end{equation}
の順に配置する.ここで LC の光学軸は $\theta=0$ において入射偏光板 POL の吸収軸と一致させ,
正面では偏光状態が Analyzer により消光される(CR$_{00}$ を保持).
斜め入射ではまず,LC により横波面内の偏光状態が回転・楕円化し,
続く $+\mathrm{A}$ により「Analyzer の吸収軸に向かう」回転成分を付与する.
最後の $\pm\mathrm{C}$ は,主として $s_3$ を抑えつつ(直線偏光化),
$(s_1,s_2)$ の方位 $\psi$ を微調整する役割を担う.
すなわち,LC と A で方位を作り,C で赤道($s_3\simeq 0$)へ押し戻しながら整列させるという役割分担である.
このとき $\phi\simeq 45^\circ$・所望の $\theta$ に対して $A$ の面内リタデーション($R_\mathrm{o}$ スケール)と
$C$ の面外リタデーション($R_\mathrm{th}$ 符号付き)を最適化すると,
「$\alpha(\theta,\phi)$ の偏位量」と「補償スタックによる $\psi(\theta,\phi)$ の回転量」が釣り合い,
$\Delta\simeq 90^\circ$ かつ $s_3\simeq 0$ を同時に満たせる.

\subsubsection{C/A/LC(LC透過軸基準)の設計思想}
透過軸基準では,LC の長軸方向を入射偏光板POL の透過軸(吸収軸から $90^\circ$)へとり,
\begin{equation}
  \mathrm{POL}\; /\; \pm \mathrm{C}\; /\; +\mathrm{A}\; /\; \mathrm{LC}\; /\; \mathrm{Analyzer}
\end{equation}
のように順序を入れ替えた C/A/LC を考える.
順序を入れ替えると,各素子が Poincar\'e 球上で与える「回転軸」と「回転順序」が変わるため,
同じ $(R_\mathrm{o},R_\mathrm{th})$ でも最終的な $\bm s$ の到達点は異なる.
しかし,設計目標は共通であり,
\begin{equation}
  s_3(\theta,\phi)\to 0,\qquad (\alpha-\psi)\bmod 180^\circ \to 90^\circ
\end{equation}
を満たすように,$+\mathrm{A}$ の $R_\mathrm{o}$ と $\pm\mathrm{C}$ の $R_\mathrm{th}$(符号を含む)を探索する.
直観的には,先段の $\pm\mathrm{C}$ が楕円率成分($s_3$)の生成/抑制に強く寄与し,
その後段の $+\mathrm{A}$ と LC が方位 $\psi$ を Analyzer の吸収軸へ引き込む.
したがって C/A/LC は「先に楕円率を制御し,後段で方位を合わせ込む」ルートとして理解でき,
LC/A/C とは異なる回転経路で同じ消光条件に到達し得る.

