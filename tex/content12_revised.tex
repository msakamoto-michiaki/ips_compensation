\section{数値モデル(IPSモードにおける光学補償最適化設計)}
\label{sec:model_details_jp}

本節では,IPS(in-plane switching)モード液晶表示の暗状態において,斜入射観察で顕在化する視野角漏れ
(off-axis light leakage)を抑制するための光学補償スタック(A-plate,C-plate,TAC 等)と液晶層を含む数値モデルを記述する.
具体的には,スタック構成(層の順序)および各層の設計パラメータ(膜厚/リタデーション,光学軸方位,必要に応じて波長依存)
を与え,視野角 $(\theta,\phi)$ で指定される各視線方向に対して,(i) 斜入射における偏光板の有効透過軸回転,
(ii) 一軸板(LC を含む)の斜入射有効遅相を取り込んだ Jones 伝搬を行い,スタック通過後の偏光状態から
暗状態漏れ透過率 $T_{\mathrm{leak}}(\theta,\phi)$,およびこれに基づくコントラスト比 $\mathrm{CR}(\theta,\phi)$ を直接評価する.
得られた $\mathrm{CR}$ 分布は等コントラスト線(isocontrast contour)として可視化でき,正面 $\mathrm{CR}_{00}$ や
視野角領域での最小値・面積指標などと併せて,補償スタック構成の比較や設計パラメータ探索・最適化の進捗を定量的に追跡できる.
先行研究では,とくに対角方向($\phi=45^\circ$)で最大化されやすい暗状態漏れを Poincar\'e 球上の幾何学として整理し,
A-plate/C-plate による補償条件(分散整合を含む)を設計する枠組みが提示されている \cite{Lee2006,Oh2015}.
本節の定式化はこれらに基づき,後段で扱う補償条件の系統導出と最適化に必要な記号・計算法・出力指標を一貫した形で与える.


\subsection{計算の概要}
計算では,まず視線方向から伝搬方向ベクトル $\bm{k}(\theta,\phi)$ を構成し,この $\bm{k}$ に直交する局所横基底を定義する.
各レイヤの光学軸は固定のグローバル直交座標系 $(x,y,z)$ 上で三次元ベクトルとして与えられるが,
複屈折による位相遅れは伝搬方向に垂直な成分により決まるため,
各レイヤに対して,$\bm{k}$ と光学軸から定まるレイヤ固有の局所基底において Jones 行列を構成する.
次に,この局所基底で表現された作用をグローバル基底(あるいは共通の横基底)へ写像し,レイヤの順序に従って行列積をとることで,
入射 Jones ベクトルから出射 Jones ベクトルを求める.
さらに必要に応じて Jones ベクトルから Stokes パラメータを算出し,偏光状態の解析(例:直線偏光方位,楕円率)にも用いる.

暗状態評価では,出射側偏光板(アナライザ)の透過軸への射影強度から漏れ透過率を求め,これをコントラスト比(Contrast ratio: CR)へ換算する

\subsection{座標系と視野角の定義}
\label{sec:coord_system_jp}
光学スタックの法線をグローバル $z$ 軸とし,基板面内に直交する $x,y$ 軸を導入する.
観察方向(視線方向)は極角 $\theta$($+z$ 軸からの傾き)と方位角 $\phi$($xy$ 平面内の角度)で表し,
対応する伝搬単位ベクトルを
\begin{equation}
\bm{k}(\theta,\phi)
=
(\sin\theta\cos\phi,\ \sin\theta\sin\phi,\ \cos\theta)^{\mathsf T}
\end{equation}
と定義する.
ここで,$\theta=0$ は正面観察に対応する.

\subsubsection{ローカル横波面と基底射影}
偏光は常に横波条件を満たすため,計算は $\bm{k}$ に垂直な 2 次元横波面上で行う.
リターダ層の光学軸(3D 単位ベクトル)を $\bm{a}$ とするとき,$\bm{k}$ に直交な面への射影
\begin{equation}
\bm{a}_{\perp}=\bm{a}-(\bm{a}\cdot\bm{k})\bm{k}
\end{equation}
を用い,正規化して
\begin{equation}
\bm{u}=\frac{\bm{a}_{\perp}}{\|\bm{a}_{\perp}\|},\qquad
\bm{v}=\frac{\bm{k}\times\bm{u}}{\|\bm{k}\times\bm{u}\|}
\end{equation}
を定義する.
$(\bm{u},\bm{v})$ は横波面上の直交基底であり,各層の Jones 表現(2 成分)を定義するための基底として用いる.
なお,$\bm{a}_{\perp}$ が数値的に極小($\bm{a}\parallel\bm{k}$ に近い)となる場合には,適切な代替基底を選択して安定化する.

\subsection{偏光板と一軸補償板のモデル}
\label{subsec:pol_ac_model}
IPS 暗状態の補償設計では,(i) 偏光板のみでも off-axis でクロスが崩れる幾何学的漏れ,
(ii) 斜入射による一軸板の有効遅相変化,(iii) それらを組み合わせたスタック全体の偏光変換,
を同一の記号体系で扱うことが重要である \cite{Lee2006,Oh2015}.
以下では O-type 偏光板(理想偏光子)と A-plate/C-plate(いずれも一軸板)のモデルをまとめる.

\subsubsection{O-type 偏光板の off-axis 有効透過軸}
\label{subsec:otype_pol}
視線方向を単位ベクトル $\hat{\bm{k}}(\theta,\phi)$ とし,偏光板の吸収軸(グローバル座標系で固定)を $\bm{c}$ とする.
O-type 偏光板では,透過する偏光方向(ordinary 方向)は $(\hat{\bm{k}},\bm{c})$ 平面に垂直であり,
\begin{equation}
\bm{o}(\hat{\bm{k}},\bm{c})
= \frac{\hat{\bm{k}}\times\bm{c}}{\|\hat{\bm{k}}\times\bm{c}\|}
\label{eq:otype_oaxis}
\end{equation}
で与えられる.
$\bm{o}$ は常に $\hat{\bm{k}}$ に直交するため,斜入射観測における偏光子/アナライザの\textbf{有効透過軸}を直接与える.
入射側偏光板の吸収軸を $\bm{c}_1$,出射側(アナライザ)の吸収軸を $\bm{c}_2$ とすると,それぞれの有効透過軸は
\begin{equation}
\bm{o}_1=\bm{o}(\hat{\bm{k}},\bm{c}_1),\qquad
\bm{o}_2=\bm{o}(\hat{\bm{k}},\bm{c}_2)
\end{equation}
である.
正面($\theta=0$)で $\bm{c}_1\perp\bm{c}_2$ と設定すると理想的に漏れはゼロとなるが,
斜入射では $\bm{o}_1,\bm{o}_2$ が $\hat{\bm{k}}$ に依存して回転し,有効なクロス条件が崩れる \cite{Lee2006}.

引用文献では,理想 O-type 偏光板のみ(光学板なし)の crossed 配置に対し,斜入射漏れが
\begin{equation}
T_{\mathrm{leakage}}
=\frac{1}{8}\,T^{4}\,
\frac{\sin^{2}2\phi\;\sin^{4}\theta_0}
{\left(1-\cos^{2}\phi\;\sin^{2}\theta_0\right)
 \left(1-\sin^{2}\phi\;\sin^{2}\theta_0\right)}
\label{eq:pol_only_leakage}
\end{equation}
と表され,特に対角方向($\phi=45^\circ$)で漏れが最大化されることが示されている \cite{Lee2006}.
ここで $T^4$ は 2 枚の偏光板の 4 界面における Fresnel 透過を表し,簡略化のため $T^4=1$ と置くことも多い.
本稿で扱う補償設計は,この幾何学的漏れを出発点とし,LC および補償板が与える偏光回転/楕円化により
off-axis においても $\bm{o}_2$ に対する消光条件を再構成することを狙う.

\subsubsection{偏光板の境界条件としての実装(入射固定と出射射影)}
\label{subsec:pol_impl}
偏光板を「吸収を伴う非ユニタリ行列」としてスタック内部に挿入すると,数値計算や記号の見通しが悪くなる.
そこで本モデルでは,crossed 偏光板を次の等価な境界条件として実装する:
\begin{itemize}
\item \textbf{入射側偏光板:} 入射電場を透過状態に固定し,
\begin{equation}
\bm{E}_0=\bm{o}_1(\hat{\bm{k}})
\end{equation}
と置く.
\item \textbf{出射側偏光板(アナライザ):} スタック通過後の電場 $\bm{E}$ をアナライザ透過状態へ射影し,
\begin{equation}
a=\bm{o}_2(\hat{\bm{k}})^{\mathsf T}\bm{E},
\qquad
T_{\mathrm{leak}}\propto |a|^2
\end{equation}
を漏れの指標とする.
\end{itemize}
この取り扱いにより,偏光板による off-axis 有効軸回転($\bm{o}_{1,2}(\hat{\bm{k}})$)と,
スタック内部の\emph{損失のない}位相遅れ(Jones 伝搬)を明確に分離できる.

\subsubsection{一軸板の斜入射有効遅相:A-plate と C-plate}
\label{subsec:retardation_model}
斜入射では,一軸板の見かけ遅相が角度依存で変化する.本稿では,先行研究に倣い,
A-plate(光学軸が面内の一軸板)および C-plate(光学軸が法線 $z$ に一致する一軸板)の遅相を
$\Gamma$(radian)として与える \cite{Lee2006}.
以下,波長を $\lambda$,膜厚を $d$,常光線・異常光線の屈折率を $n_o,n_e$ とする.

\paragraph{A-plate(面内一軸)の遅相}
A-plate の光学軸方位と入射方位との差を相対方位角 $\phi_{\mathrm{rel}}$ とすると,
\begin{align}
\Gamma_{A}(\theta,\phi_{\mathrm{rel}})
&= \frac{2\pi}{\lambda} d
\Biggl[
n_e \left(1-\frac{\sin^2\theta\,\sin^2\phi_{\mathrm{rel}}}{n_e^2}
          -\frac{\sin^2\theta\,\cos^2\phi_{\mathrm{rel}}}{n_o^2}\right)^{1/2}
-
n_o \left(1-\frac{\sin^2\theta}{n_o^2}\right)^{1/2}
\Biggr]
\label{eq:GammaA}
\end{align}
で与えられる \cite{Lee2006}.
数値計算では平方根内部が負にならないよう,微小クリップ等により安定化する.

\paragraph{C-plate(法線一軸)の遅相}
C-plate は光学軸が法線 $z$ に一致する一軸板であり,理想的には方位角に依存せず $\theta$ のみの関数となる:
\begin{align}
\Gamma_{C}(\theta)
&= \frac{2\pi}{\lambda}\frac{d}{\cos\theta}
\left[
\left(
\frac{n_o^{2}n_e^{2}}{n_o^{2}\sin^{2}\theta + n_e^{2}\cos^{2}\theta}
\right)^{1/2}
-
n_o
\right].
\label{eq:GammaC}
\end{align}
なお,$\Delta n\equiv n_e-n_o$ の符号を反転させることは,Poincar\'e 球上の回転向きを反転させる自由度に対応する.
補償設計・最適化ではこの符号(正負)も有効な探索自由度となり得る \cite{Oh2015}.

\subsection{スタックの Jones 伝搬と漏れ評価}
\label{subsec:jones_stack}

\subsubsection{ローカル基底における retarder 演算子}
各層 $n$ に対し,$(\bm{u}_n,\bm{v}_n)$ 基底で電場を 2 成分 Jones ベクトル
$\bm{E}_n=(E_{u,n},E_{v,n})^{\mathsf T}$ として表す.
遅相 $\Gamma_n$ を持つ理想 retarder の Jones 行列は
\begin{equation}
\bm{J}(\Gamma_n)=
\begin{pmatrix}
e^{+i\Gamma_n/2} & 0 \\
0 & e^{-i\Gamma_n/2}
\end{pmatrix}
\end{equation}
であり,基底変換(同一横波面上の回転)を介してスタック全体の伝搬を記述する.
実装上は,グローバル 3D 電場を横波面へ射影したうえで,各層の基底へ回転し,$\bm{J}$ を作用させて戻す,
という等価な手続きで表現できる(ローカル基底に結び付いた \emph{Jones-like} 伝搬).

\subsubsection{全体の漏れ振幅とコントラスト指標}
偏光板を境界条件として含めると,視野方向 $(\theta,\phi)$ における漏れ振幅は
\begin{equation}
a(\theta,\phi)=
\bm{o}_2(\hat{\bm{k}})^{\mathsf{T}}
\left(\prod_{n}\bm{M}_n(\hat{\bm{k}},\bm{a}_n,\Gamma_n)\right)
\bm{o}_1(\hat{\bm{k}}),
\qquad
T_{\mathrm{leak}}\propto |a|^2
\end{equation}
と書ける.ここで $\bm{M}_n$ は層 $n$ の 3D→横波面→3D を含む有効線形作用素である(損失なし).
暗状態のコントラスト比は,白状態透過率 $T_{\mathrm{white}}$ を用いて
$\mathrm{CR}=T_{\mathrm{white}}/T_{\mathrm{leak}}$
と定義するのが一般的であり,簡略化として $T_{\mathrm{white}}\approx 1$ を置けば
$\mathrm{CR}\approx 1/T_{\mathrm{leak}}$ として扱える.

\subsection{Stokes 表現と Poincar\'e 球による解釈}
\label{subsec:matrix_to_stokes}
横波面内の直交基底 $(\bm{u},\bm{v})$ 上で Jones ベクトルを $\bm{E}=(E_u,E_v)^{\mathsf T}$ と表すと,
Stokes パラメータ $(S_0,S_1,S_2,S_3)$ は
\begin{align}
S_0 &= |E_u|^2+|E_v|^2,\qquad
S_1 = |E_u|^2-|E_v|^2, \nonumber\\
S_2 &= 2\,\mathrm{Re}(E_uE_v^\ast),\qquad
S_3 = 2\,\mathrm{Im}(E_uE_v^\ast),
\end{align}
で定義される.規格化 Stokes ベクトル $\bm{s}=(S_1/S_0,S_2/S_0,S_3/S_0)$ は Poincar\'e 球上の一点に対応し,
各層通過に伴う偏光状態の変化を幾何学的に追跡できる.
先行研究では,対角方向で最大化する漏れや,波長分散(RGB 間での偏光状態のずれ)を,
Poincar\'e 球上の回転軌跡の観点から設計する方法が提示されている \cite{Lee2006,Oh2015}.
本モデルでも同様に,幾何学的要因(偏光板有効軸の回転)と,
LC/補償板による偏光回転・楕円化を同一表現で比較できる.

\subsection{設計指標と最適化問題の定式化}
\label{subsec:ips_comp_design_philosophy}
補償設計の目的は,正面コントラスト $\mathrm{CR}_{00}$ を維持しつつ,視野角全域にわたる暗状態漏れを低減することである.
そのため,設計変数 $\bm{p}$(各層の遅相,光学軸方位,波長分散パラメータなど)に対し,
例えば次のような制約付き最適化として定式化できる:
\begin{align}
\text{minimize}\quad & 
\mathcal{J}(\bm{p}) = 
\int_{\Omega} w(\theta,\phi)\,g\!\left(T_{\mathrm{leak}}(\theta,\phi;\bm{p})\right)\,d\Omega
+ \alpha\,\Pi\!\left(\mathrm{CR}_{00}(\bm{p})-\mathrm{CR}_{00}^{\min}\right), \\
\text{subject to}\quad &
\bm{p}\in \mathcal{P},
\end{align}
ここで $\Omega$ は評価する視野角領域,$w$ は重み関数,$g$ は漏れに対する評価関数(例:$g(x)=x$,$g(x)=\log x$),
$\Pi$ は制約違反に対するペナルティ,$\mathcal{P}$ は実現可能な設計空間(膜厚範囲,材料分散タイプなど)である.
実務上は,等コントラスト線(isocontrast contour)を可視化し,
$\mathrm{CR}_{00}$,視野角領域での $\min \mathrm{CR}$,あるいは $\mathrm{CR}>\mathrm{thr}$ を満たす面積比などの
スカラー指標を併用して設計の進捗を追跡する.
さらに白色表示の暗状態を目的とする場合は,RGB 代表波長での評価を組み合わせ,
Poincar\'e 球上での偏光状態の集約(dispersion elimination)を指針として用いる \cite{Lee2006,Oh2015}.

% --- bibliography (example; move to your main .bib or thebibliography) ---
% \begin{thebibliography}{9}
% \bibitem{Lee2006} J.-H. Lee, H. Choi, S. H. Lee, J. C. Kim, and G.-D. Lee,
% ``Optical configuration of a horizontal-switching liquid-crystal cell for improvement of the viewing angle,''
% Appl. Opt. \textbf{45}, 7280--7285 (2006).
% \bibitem{Oh2015} S.-W. Oh \emph{et al.},
% ``Optical compensation methods for the elimination of off-axis light leakage in an in-plane switching liquid crystal display,''
% J. Inf. Disp. (year/volume/pages; see attached PDF).
% \end{thebibliography}
