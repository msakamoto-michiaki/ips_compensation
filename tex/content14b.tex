\section{Case-1 における光学軸配置(プログラム実装の確認)}
本節では,Case-1(LC/A/C,LC吸収軸基準)で用いている光学軸の定義と,
実装(プログラム)上の角度依存のしかたを整理する。
以降,試料面内の基準座標を $(x,y)$,法線方向を $z$ とし,
方位角 $\alpha$ は $x$ 軸から反時計回り($+z$ まわり回転)に測る。

\subsection{偏光板(O-type)の吸収軸・透過軸の定義}
本実装では偏光板はスタック(retarder列)には含めず,
入力側偏光板(POL)およびアナライザ(ANL)の吸収軸ベクトル $\bm{c}_1,\bm{c}_2$
として別途与える(O-type の定義).
基準状態(回転角 $0^\circ$)では
\begin{equation}
\bm{c}_1 = \hat{\bm{x}},\qquad
\bm{c}_2 = \hat{\bm{y}},
\end{equation}
すなわち「入力側吸収軸が $x$,出射側(アナライザ)吸収軸が $y$」のクロスニコルである。
入力側・出射側の独立回転角をそれぞれ \texttt{pol\_in}, \texttt{pol\_out} とすると,
\begin{equation}
\bm{c}_1(\texttt{pol\_in})=\mathrm{R}_z(\texttt{pol\_in})\,\hat{\bm{x}},\qquad
\bm{c}_2(\texttt{pol\_out})=\mathrm{R}_z(\texttt{pol\_out})\,\hat{\bm{y}}.
\end{equation}
対応する方位角(吸収軸)は
\begin{equation}
\alpha_{\mathrm{pol,in}}^{(\mathrm{abs})}=\texttt{pol\_in},\qquad
\alpha_{\mathrm{pol,out}}^{(\mathrm{abs})}=90^\circ+\texttt{pol\_out},
\end{equation}
透過軸は吸収軸から $+90^\circ$ 回転した方向であるから
\begin{equation}
\alpha_{\mathrm{pol,in}}^{(\mathrm{tran})}=90^\circ+\texttt{pol\_in},\qquad
\alpha_{\mathrm{pol,out}}^{(\mathrm{tran})}=\texttt{pol\_out}.
\end{equation}

\subsection{Case-1(LC/A/C,LC吸収軸基準)のスタックと角度パラメータ}
Case-1 ではスタックを
\[
\mathrm{LC}\;/\;\mathrm{A}\;/\;\mathrm{C}
\]
の順に配置する(C-plate は $z$ 軸まわり等方で,光学軸は $\hat{\bm{z}}$ に固定).
ここで重要なのは,\textbf{LC は入力側基準(\texttt{pol\_in})に追随する一方,
A-plate は出射側基準(\texttt{pol\_out})に追随する}ように実装されている点である。

LC吸収軸基準(\texttt{lc\_basis="abs"})では,LC の基準方位角を $0^\circ$($x$ 方向)に置き,
入力側回転 \texttt{pol\_in} と LC の相対角 \texttt{relLC} を加えて
\begin{equation}
\alpha_{\mathrm{LC}} = 0^\circ + \texttt{pol\_in} + \texttt{relLC}
\label{eq:alpha_LC_case1}
\end{equation}
とする。

一方 A-plate は「上側A(upper)相当」を用い,基準方位角を
\texttt{A\_base}=$90^\circ$($y$ 方向)に固定した上で,
出射側回転 \texttt{pol\_out} と A の相対角 \texttt{relA} を加えて
\begin{equation}
\alpha_{\mathrm{A}} = 90^\circ + \texttt{pol\_out} + \texttt{relA}
\label{eq:alpha_A_case1}
\end{equation}
とする。

C-plate は一軸性で光学軸を
\begin{equation}
\bm{a}_{\mathrm{C}}=\hat{\bm{z}}
\end{equation}
に固定し,面内方位角は持たない(方位角最適化の自由度は無い)。

\begin{table}[t]
  \centering
  \caption{Case-1(LC/A/C,LC吸収軸基準)における角度パラメータと方位角の定義。}
  \label{tab:angles_case1}
  \begin{tabular}{lll}
    \hline
    記号(実装変数) & 物理的意味 & 方位角($x$ 基準)\\
    \hline
    \texttt{pol\_in}  & 入力側偏光板(吸収軸)回転 & $\alpha_{\mathrm{pol,in}}^{(\mathrm{abs})}=\texttt{pol\_in}$ \\
    \texttt{pol\_out} & 出射側偏光板(吸収軸)回転 & $\alpha_{\mathrm{pol,out}}^{(\mathrm{abs})}=90^\circ+\texttt{pol\_out}$ \\
    \texttt{relLC}    & LC の入力側基準からの相対角 & $\alpha_{\mathrm{LC}}=\texttt{pol\_in}+\texttt{relLC}$ \\
    \texttt{relA}     & A の出射側基準からの相対角  & $\alpha_{\mathrm{A}}=90^\circ+\texttt{pol\_out}+\texttt{relA}$ \\
    \hline
  \end{tabular}
\end{table}

\paragraph{具体例(入力側のみ回転させる場合の挙動)}
例えば Analyzer 側を基準に固定し(\texttt{pol\_out}=0),
\texttt{relA}=$0.25^\circ$, \texttt{relLC}=$0.50^\circ$ とした場合,
\begin{equation}
\alpha_{\mathrm{LC}}=\texttt{pol\_in}+0.50^\circ,\qquad
\alpha_{\mathrm{A}}=90.25^\circ,\qquad
\alpha_{\mathrm{pol,out}}^{(\mathrm{abs})}=90^\circ.
\end{equation}
このように Case-1 の実装では,\texttt{pol\_in} を掃引すると LC の方位角のみが追随し,
A-plate の方位角は(\texttt{pol\_out} を固定している限り)一定に保たれる。
したがって
「\texttt{pol\_in} の回転に対して LC と A が同時に追随する」
という挙動にはなっていない点に注意する。

\subsection{膜厚(リタデーション)パラメータ化:A-plate と C-plate}
等方屈折率を $n_o$,複屈折を $\Delta n$ とし,異常屈折率を $n_e=n_o+\Delta n$ とする。
位相差(リタデーション)$\mathrm{Re}$ と膜厚 $d$ の関係は
\begin{equation}
  \mathrm{Re} = \Delta n \, d
\end{equation}
であり,
\begin{equation}
  d\,[\mu\mathrm{m}] = \frac{\mathrm{Re}\,[\mathrm{nm}]}{1000\,\Delta n}
  \label{eq:d_from_Re_general}
\end{equation}
と換算できる。

\paragraph{A-plate(1枚)の扱い}
Case-1 の最適化では A-plate を 1 枚のみ用い,そのリタデーションは
\begin{equation}
\mathrm{Re}_{\mathrm{A}} = A_{\mathrm{scale}}\;\mathrm{Re}_{\mathrm{A,base}}
\end{equation}
としてスケール係数 \texttt{A\_scale} を探索する。
膜厚は採用する $\Delta n_A$(\texttt{A\_kind}=\texttt{upper}/\texttt{lower})に応じて
\begin{equation}
d_{\mathrm{A}}=\frac{\mathrm{Re}_{\mathrm{A}}}{\Delta n_A}
\end{equation}
で与える。
なお,\texttt{Re\_A,base} の具体値(例えば $\mathrm{Re}_{\mathrm{LC}}/2$ を採る等)は
\textbf{設計基準として外部から与えるべき定数}であり,
A-plate が 1 枚の設計に合わせる場合は
$\mathrm{Re}_{\mathrm{A,base}}$ の定義自体(および探索範囲)を明確にした上で記述する。

\paragraph{C-plate の扱い(signed ReC)}
C-plate は \texttt{ReC\_nm}(nm)を直接探索パラメータとし,
\eqref{eq:d_from_Re_general} により膜厚を決定する。
また \texttt{ReC\_nm} の符号は「正負C(複屈折符号)」として扱い,
\[
d_{\mathrm{C}} = \frac{|\texttt{ReC\_nm}|}{1000\,|\Delta n_{\mathrm{C}}|},\qquad
\Delta n_{\mathrm{C,eff}}=\mathrm{sign}(\texttt{ReC\_nm})\,\Delta n_{\mathrm{C}}
\]
のように,厚みは正で保持しつつ有効複屈折の符号を反転させる実装になっている。

\subsection{Case-1 結果ファイル(\texttt{case1\_results\_20260108\_152708.txt})との対応}
\texttt{case1\_results\_20260108\_152708.txt} は,
\texttt{progress\_case1.csv} の最終行(\texttt{ROW=-1})を読み出し,
指定した視野角(例:$\theta=30^\circ,\phi=45^\circ$)において
Stokes パラメータ($s_1,s_2,s_3$)をステージ(例:\texttt{el\#2\_C})で評価するための
後処理スクリプトである。
ここでは偏光板は \texttt{pol\_in=pol\_out=0} の実験室基準で固定し,
Stokes 基底は \texttt{basis="pol\_in"} を用いているため,
本節で定義した角度テーブル(Table~\ref{tab:angles_case1})と整合する。

% 図:Case-1 最適解の ISO-CR 等高線図(out_opt/iso_1_best.png)をここに挿入
% \begin{figure}[t]
%   \centering
%   \includegraphics[width=0.85\linewidth]{figs/iso_1_best.png}
%   \caption{Case-1(LC/A/C, LC吸収軸基準)の最適解に対する CR 等高線図(polar 表示).}
%   \label{fig:iso_case1_best}
% \end{figure}
